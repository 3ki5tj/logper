\documentclass{ws-ijbc}
\usepackage{multirow}
\usepackage{centernot}
%\usepackage{hyperref}
\usepackage{graphicx}
\usepackage{epstopdf}
\usepackage{calc}
\usepackage{cancel} % strike out (cancel) in math mode
\usepackage{ws-rotating}
\begin{document}



\newcommand{\odd}{\mathrm{odd}}

% bold font for vectors
\newcommand{\vct}[1]{\mathbf{#1}}
% the cycle and sum operator
%\newcommand{\C}{\mathcal{C}}


% center column in tabularx
\newcolumntype{C}{>{\centering\arraybackslash}X}
\newcolumntype{L}{>{\raggedright\arraybackslash}X}
\newcolumntype{R}{>{\raggedleft\arraybackslash}X}


\newcommand{\vx}{\vct x}
\newcommand{\vxn}[1]{\vx^{(#1)}}
\newcommand{\Tr}{\mathrm{Tr}}
% basis set of cyclic polynomials
\newcommand{\Pset}{\mathcal P}
\newcommand{\NB}{N_\Pset}
\newcommand{\Rlam}{(R, \lambda)}

%\newcommand{\eq}{}
%\newcommand{\eqs}{}
\newcommand{\eq}{Eq.}
\newcommand{\eqs}{Eqs.}
\newcommand{\req}[1]{(\ref{eq:#1})}
\newcommand{\refeq}[1]{\eq\,\req{#1}}
\newcommand{\refeqs}[1]{\eqs\,\req{#1}}
\newcommand{\reqsub}[2]{(\ref{eq:#1}#2)}
\newcommand{\refeqsub}[2]{\eq\,\reqsub{#1}{#2}}
\newcommand{\refeqssub}[2]{\eqs\,\reqsub{#1}{#2}}

\newcommand{\refthm}[1]{Theorem \ref{thm:#1}}
\newcommand{\refthms}[1]{Theorems \ref{thm:#1}}
\newcommand{\refsec}[1]{Section \ref{sec:#1}}
\newcommand{\refsecs}[1]{Sections \ref{sec:#1}}
\newcommand{\refapd}[1]{Appendix \ref{apd:#1}}
\newcommand{\reftab}[1]{Table \ref{tab:#1}}
\newcommand{\reftabs}[1]{Tables \ref{tab:#1}}
\newcommand{\reffig}[1]{Fig. \ref{fig:#1}}
\newcommand{\reffigs}[1]{Figs. \ref{fig:#1}}

% ws-ijbc doesn't like the following lines
%\newtheorem{theorem}{Theorem}
%\newenvironment{remark}[1][1]%
%{\par\noindent\textbf{Remark #1.} }{\medskip}


\catchline{}{}{}{}{}




\title{Cycles of the logistic map}
\author{Cheng Zhang}
%\affiliation{Applied Physics Program and Department of Bioengineering, \
%Rice University, Houston, TX 77005}
\address{Applied Physics Program \& Department of Bioengineering,
Rice University, Houston, TX 77005, USA}

\maketitle


\begin{history}
\received{}
\end{history}


\begin{abstract}
%\abstract{
%
The onset and bifurcation points of the $n$-cycles of
  a polynomial map are located
  through a characteristic equation
  connecting the cyclic polynomials of the cycle points.
The minimal polynomials of the critical parameters
  of the logistic, H\'enon, and cubic maps are obtained
  for $n$ up to 13, 9, and 8,
  respectively.
%
%}
\end{abstract}

\keywords{Logistic map; H\'enon map; cubic map; cycles; exact polynomials.}


%\maketitle

\twocolumn
%
%
\section{Introduction}
%
%

Consider the logistic map \cite{may, strogatz}:
%
\begin{equation}
  x_{k+1} = f(x_k) \equiv r \, x_k \, ( 1 - x_k ).
\label{eq:logmap}
\end{equation}
%
The iterated sequence $x_1$,
  $x_2 = f(x_1)$,
  $x_3 = f(x_2)$, $\ldots$
can be conveniently visualized on the cobweb plot,
  as shown in \reffig{cobweb}.
%
Starting from $(x_1, x_1)$ on the diagonal,
  each vertical arrow takes $(x_k, x_k)$ to $(x_k, y)$,
  where $y = f(x_k) = x_{k+1}$;
the next horizontal arrow then reflects $(x_k, y)$ to
  $(y, y) = (x_{k+1}, x_{k+1})$,
  where the next iteration starts.
%
%
\begin{figure}[h]
  \begin{center}
  \begin{minipage}{\linewidth}
        \includegraphics[angle=-90, width=\linewidth]{cobweb.pdf}
  \end{minipage}%
  \end{center}
  \caption{\label{fig:cobweb}
  Cobweb plot of the logistic map.}
\end{figure}
%
%
After many iterations, there can be three ultimate patterns:
  (i) a \emph{fixed point} or a constant (including infinity),
      e.g., \reffig{cobweb}(a);
  (ii) a periodic \emph{cycle} or a self-repeating loop,
      e.g., \reffigs{cobweb}(b)-(e);
or
  (iii) a chaotic trajectory,
      e.g., \reffig{cobweb}(f).
We will focus on the first two cases here.


A fixed point is a solution of $x^* = f(x^*)$.
%
It is stable if $x_1$ deviates slightly from $x^*$
  and the sequence still converges to $x^*$.
%
For a differentiable $f$,
  a stable fixed point requires $|f'(x^*)| \le 1$
  to reduce deviations in successive iterations \cite{strogatz}.


An $n$-cycle is a sequence $x_1, \dots, x_n$,
  such that $n$ is the smallest positive integer
  that satisfies $x_1 = x_{n+1} = f^n(x_1)$,
  where $f^n$ is the $n$th iterate of $f$,
  e.g., $f^3(x) = f(f(f(x)))$.
Obviously, any $x_k$ in an $n$-cycle of $f$ is a fixed point of $f^n$
%
\big[but the converse is untrue, for a fixed point of $f^n$
  can also be a fixed point of $f^d$ for a divisor $d$ of $n$:
  if $f^d(x) = x$, then $f^n(x) = f^d(\cdots f^d(x)\cdots) = x$\big].
%
%
Thus, a cycle can be classified as stable or unstable
  according to the corresponding fixed point of $f^n$:
  a stable cycle requires
  $\big| \frac {d} {dx} f^n(x_1) \big| \le 1$,
  or by the chain rule,
%
%
%
\begin{equation}
  \Big| f'(x_1) \cdots f'(x_n) \Big| \le 1,
\label{eq:der}
\end{equation}
%
%
The onset and bifurcation points
  are defined at the loci
  where $\frac {d} {dx} f^n(x_1)$ reaches $+1$ and $-1$,
  respectively \cite{strogatz}.
%
%
%
The cycle and its stability depend on the parameter $r$.
%
Our purpose is to present an algorithm to identify
  all regions of $r$ that allow stable $n$-cycles.








\section{\label{sec:logmap}Logistic map}


We will illustrate the algorithm on the logistic map \cite{may, strogatz}
  defined in \refeq{logmap}.
If $r$ is real,
  we will find windows $(r_a, r_b)$,
    within which stable $n$-cycles can exist.
%
%The window boundaries can be classified
%  to onset and bifurcation points according to \refeq{der};
Here, if $r > 0$, then
  $r_a$ ($r_b$) are the onset (bifurcation) points.
%
There are generally multiple such windows for a single $n$,
  but all onset points satisfy the same polynomial equation,
  and all bifurcation points another.
%
Our goal is thus to find the two polynomials for a given $n$.


To simplify the calculation, we first change variables \cite{mandelbrot, brown1, brown2, saha} as
%
%\begin{equation}
\[
    x^{\mathrm{new}} \equiv r(x^{\mathrm{old}} - 1/2),
    \quad R \equiv r(r-2)/4,
\]
%\label{eq:chvars}
%\end{equation}
%
and rewrite the map, in terms of $x^{\mathrm{new}}$, as
%
\begin{equation}
  x_{k+1} = f(x_k) \equiv R - {x_k}^2.
  %\tag{$\ref{eq:logmap}'$}
\label{eq:logmaps}
\end{equation}
%
%\refeq{logmaps} simplifies \refeq{logmap} as $x_k$ now only occurs once.
%
We will solve the cycle boundaries as zeros of some polynomials of $R$,
and the corresponding polynomials for $r$ can be obtained by $R \rightarrow r(r-2)/4$.
%
%the numerical $r$ values can always be obtained from $r = 1 + \sqrt{1 + 4R}$.
%






\subsection{Overall plan}

Since the $n$-cycles constitute a subset of the fixed points of $f^n$,
we will first find
  the polynomials at the stability boundaries
  for the fixed points of $f^n$
  (\refsecs{cyclic} to \ref{sec:examples}).
The contributions from shorter $d$-cycles ($d|n$)
  are lower-degree factors of the polynomials,
  and can be readily removed (\refsec{primfac}).


At the first glance, the polynomials of the fixed points of $f^n$
  can be obtained by brute force.
We can solve
  \refeqs{logmaps} and express $x_1, \ldots, x_n$
  in terms of $R$,
  and then plug the solution into \req{der}.
The result would contain $R$ only,
  and is therefore the answer.
But since \refeqs{logmaps} are nonlinear,
  it is nontrivial to reduce the final equation of $R$
  into a polynomial one.
Nonetheless, on a computer, one can
  construct a polynomial Gr\"obner basis \cite{kk1}
  that gradually eliminates $x_k$ out of the $n+1$ equations.
As a special case, if $x_2, \dots, x_n$ are eliminated first,
  the final elimination of $x_1$ can be accomplished
  by setting the resultant of $f^n(x_1) = x_1$ and
  \refeq{der} or $\frac{d}{dx}f^n (x_1) = \pm 1$
  to zero \cite{burm},
  which furnishes the needed polynomial equation of $R$.
%
These approaches, however, do not
  exploit the cyclic structure of \refeqs{logmaps},
  and can thus be improved by the following alternative.


Instead of solving \refeqs{logmaps} for $x_k$,
  we will derive a set of \emph{homogeneous linear} equations
  $\sum_j A_{ij} C_j = 0$
  regarding cyclic polynomials $\{C_j\}$ of $x_k$
  (an example of a cyclic polynomial is $x_1 x_2 + x_2 x_3 + \dots + x_n x_1$).
%
Now the matrix formed by the coefficients $A_{ij}$ of the equations
  must have a zero determinant,
  for the cyclic polynomials $\{C_j\}$ are not zeros altogether.
%
%As the coefficients are polynomials of $R$ only,
Thus, the zero-determinant condition $|\mathbf A| = 0$ gives
  the needed polynomial equation of $R$,
whose roots contain all fixed points of $f^n$.
%
%Note that we will obtain different equations
%  at the onset and bifurcation points
%  because the coefficients,
%  hence the matrices and determinants,
%  will be different.
%


\subsection{\label{sec:cyclic}Cyclic polynomials}


A cyclic polynomial is an invariant
  under the cycling of variables
  $x_1 \rightarrow x_2, x_2 \rightarrow x_3,
  \ldots, x_n \rightarrow x_1$,
  e.g., $a = x_1 x_2 + x_2 x_3 + x_3 x_4 + x_4 x_1$
  and $b = x_1 x_3 + x_2 x_4$ (for $n = 4$).
%
It is, however, not necessarily symmetric
  (i.e., invariant under the exchange of any two $x_k$ and $x_{j}$),
  e.g., $a$ and $b$ are not symmetric,
  but $a + b$ is.


A cyclic polynomial can be generated by summing over
  distinct cyclic versions of a simpler polynomial of $x_k$, or a \emph{generator}.
For example,
$x_1 x_2$ is a generator of $a$;
$x_1 x_3$ is that of $b$;
and
$x_1 x_2 + \frac{1}{2} x_1 x_3$ is that of $a + b$.
Note that the coefficient before $x_1 x_3$
  is 1 in the second case
  for there are only two distinct versions,
  but $\frac{1}{2}$ in the third case
  for there are four. % distinct versions of the generator.



Cyclic polynomials
  generated from monomials of unit coefficient
  (such as $a$ and $b$, but not $a + b$ in the above example)
  can be labeled as follows.
We write the monomial generator as
  ${x_1}^{e_1} {x_2}^{e_2} \dots {x_n}^{e_n}$,
  and form a sequence $p$ of indices with
  $e_1$ 1's, $e_2$ 2's, \ldots, $e_n$ $n$'s.
The sequence $p$ serves as the label of the cyclic polynomial $C_p(\vxn{n})$,
  where $\vxn{n} \equiv \{x_1, \ldots, x_n\}$
[we will, however, drop the superscript ``$^{(n)}$'' or
  even the argument ``$(\vxn{n})$'' entirely when it is clear].
For example,
  $C_{12}(\vx)  = x_1 x_2 + x_2 x_3 + \dots + x_n x_1$ and
  $C_{112}(\vx) = {x_1}^2 x_2 + {x_2}^2 x_3 + \dots + {x_n}^2 x_1$.
%The cycle length $n$ is omitted for convenience.
%  for we will mostly work with a fixed $n$ at a time.
In case the cyclic polynomial has several generators,
%  e.g., both $x_1 x_2$ and $x_2 x_3$ are generators of $C_{12}(\vx)$
%   (assuming $n \ge 3$),
  we pick the one with the smallest $p$
  in the sense of lexicographic order,
  e.g., we choose $x_1 x_2$ instead of $x_2 x_3$ for $C_{12}(\vx)$
  because $12 < 23$ ($n \ge 3$).
Finally, we add $C_0(\vx) \equiv 1$ for completeness.
%
%
%




\subsection{Square-free reduction}




A cyclic polynomial is \emph{square-free}
  if it has no square or higher powers of $x_k$,
e.g., $C_{12}(\vx) = x_1 x_2 + \cdots$ is square-free,
  $C_{112}(\vx) = {x_1}^2 x_2 + \cdots$ is not.
%see \reftab{sqrfreepoly} for more examples.
The label $p$ of a square-free polynomial
  has no repeated index.
The set of all square-free $p$ is denoted by
$\Pset = \{0, 1, 12, 13, \ldots, 123, 124, \ldots, 12\dots n\}$,
and its size $|\Pset|$ equals the number $\NB$ of
  square-free cyclic polynomials.
%
%
The set of square-free cyclic polynomials
  furnishes a basis for expanding cyclic polynomials:



\begin{theorem}
  For the logistic map \refeq{logmaps},
  any cyclic polynomial $K(\vx)$
  formed by the $n$-cycle points
  $\vx = \{x_1, \ldots, x_n\}$
  is a linear combination of
  the square-free cyclic polynomials $C_p(\vx)$:
\[
  K(\vx) = \sum_{p \in \Pset} f_p(R) \, C_p(\vx),
\]
  where $\Pset$ % = \{0, 1, 12, 13, \ldots, 12\dots n\}$ is
  is the set of indices of all square-free cyclic polynomials,
  and $f_p(R)$ are polynomials of $R$.
  \label{thm:sqrfree}
\end{theorem}
%
%
\begin{proof}
Consider the following square-free reduction.
%
Given a cyclic polynomial $K(\vx)$,
  we recursively apply \refeq{logmaps} as
  $x_k^2 \rightarrow R - x_{k+1}$,
  until all squares or higher powers of $x_k$ are eliminated.
The process will not last indefinitely
  for each substitution reduces the degree in $x_k$ %(no matter which $k$)
  by one.
Since the original polynomial is cyclic,
  so is the final square-free polynomial.
Terms involving no $x_k$
  are collected to serve as the coefficient before $C_0(\vx) \equiv 1$.
%Since no square or higher powers of $x_k$
%  can survive the reduction,
%  the final cyclic polynomials are square-free;
The coefficients $f_p$ are polynomials of $R$,
  for $R$ is the only variable introduced by the substitutions.
\end{proof}
%
%
For example, the cyclic polynomial
  $K(\vxn{2}) = {x_1}^2 \, x_2 + {x_2}^2 \, x_1$
can be written as $K = (R + 1) \, C_{1} - 2 R \, C_0$
for
${x_1}^2 \, x_2 = R \, x_2 - {x_2}^2 = R \, x_2 - R + x_1$ and
${x_2}^2 \, x_1 = R \, x_1 - R + x_2$.


\refthm{sqrfree}
  shows that any cyclic polynomial can be expanded
  as a combination of the square-free ones.
Below we show that at the onset and bifurcation points,
  the square-free cyclic polynomials $C_p$ are themselves
  linearly connected by $\NB$ homogeneous equations.
The matrix of the connecting coefficients then must have a zero determinant,
  which yields the desired polynomial.


\subsection{\label{sec:algo}Fixed points of $f^n$}


We first observe that the derivative of $f^n$
is a cyclic polynomial:
%
%
\begin{equation}
  \Lambda(\vx)
   = f'(x_1) \dots f'(x_n)
   = (-2)^n x_1 \dots x_n.
\label{eq:logder}
\end{equation}
%
%
Since $C_p(\vx)$ is also cyclic,
  so is $\Lambda(\vx) \, C_p(\vx)$,
The latter can be expanded by \refthm{sqrfree} as
\begin{equation}
  \Lambda(\vx) \, C_p(\vx) = \sum_{q \in \Pset} T_{pq}(R) \, C_q(\vx),
\label{eq:xcp0}
\end{equation}
where $T_{pq}(R)$ is a polynomial of $R$,
and $p, q \in \Pset$.



By \refeq{der},
  $\Lambda(\vx)$ equals to $\lambda = +1$ or $-1$
  at the onset or bifurcation point,
  respectively;
  so \refeq{xcp0}
  becomes a homogeneous linear equation of
  $C_p(\vx)$:
\begin{equation}
  \lambda \, C_p(\vx) = \sum_{q \in \Pset} T_{pq}(R) \, C_q(\vx),
\label{eq:xcp}
\end{equation}
%
%
%where $\lambda$ is $+1$ and $-1$ at the onset and bifurcation points, respectively;
or in matrix form,
\begin{equation}
  \big[ \lambda \, \vct I - \vct T(R) \big] \, \vct C = 0,
\tag{\ref{eq:xcp}'}
\end{equation}
%
%
where
$\vct I$ is the $\NB \times \NB$ identity matrix,
$\vct T(R) = \{T_{pq}(R)\}$ is an $\NB \times \NB$ matrix,
and
$\vct C = \{C_p(\vx)\}$ is an $\NB$-dimensional column vector.




Since a set of homogeneous linear equations
  has a non-trivial solution
  only if the determinant of the coefficient matrix
  is zero, we have
\begin{equation}
  A_n\Rlam \equiv \big| \lambda \, \vct I - \vct T(R) \big| = 0.
\label{eq:xr}
\end{equation}
Here, $A_n\Rlam$ is a polynomial of $R$ and $\lambda$,
  and the subscript $n$ is attached such that
  it can be used later with $A_d\Rlam$ of divisors $d$ of $n$.
\refeq{xr} is a necessary condition
  since $C_p(\vx)$ cannot vanish altogether;
and since it involves $R$ only,
  the polynomial expansion of the determinant
  gives the answer to our problem.
%Note that we did not solve the values $C_p$ in the process.


To summarize, we have
\begin{theorem}
  At the onset and bifurcation points,
  the square-free cyclic polynomials $C_p(\vx)$
  of the cycle points $\vx = \{x_k\}$
  are linear related by % \refeq{xcp},
  $\lambda \, C_p(\vx) = \sum_{q \in \Pset} T_{pq}(R) \, C_q(\vx)$,
  with $\lambda$ being $+1$ and $-1$, respectively,
  and $T_{pq}(R)$ being the coefficients
  from the square-free reduction of the cyclic polynomial
  $C_p(\vx) \, \prod_k f'(x_k)$. %[\refeq{logder}].
  Thus, the fixed point of $f^n$
  is marginally stable when $R$ satisfies
  characteristic equation $\big| \lambda \, \vct I - \vct T(R) \big| = 0$.
  \label{thm:main}
\end{theorem}

\begin{remark}
  Generally, the algorithm can trace a contour of marginal stability of $R$
  in the complex plane
  for complex cycles $\vx$
  if $\lambda$ is set to $\exp(i\theta)$ with $\theta \in [0, 2\pi]$.
  The region enclosed by the contour corresponds to
    a bulb in the inverted Mandelbrot set
  (the correspondence follows from the transformation $z_k = -x_k, c = -R$, which
  reduces \refeq{logmaps} to the quadratic map $z_{k+1} = c + {z_k}^2$
  used in the Mandelbrot set \cite{mandelbrot}),
  see ref. \cite{stephenson2, stephenson3}. % and \reffig{logbulb}.
\end{remark}


\begin{remark}
  With $\lambda = 0$,
  the algorithm yields the polynomial of the superstable point,
  at which the initial deviation from a cycle point vanishes
  to the linear order after $n$ iterations of $f$.
  But we have a better algorithm in this case:
  since at least one of the $x_k$ is zero by \refeq{logder},
  then $f^n(x_k = 0) = x_{n+k} = 0$, which provides
  the desired equation of $R$ \cite{strogatz}.
\end{remark}





\subsection{\label{sec:examples}Examples}




We illustrate the above algorithm on several small-$n$ cases.
%
It is still helpful to have a mathematical software
  in order to verify certain steps
  (e.g., in computing the determinants and their factorization).



For $n = 1$, we have
  $C_0 = 1$, $C_1 = x_1$, $\Lambda = -2 \, x_1$, and
%Thus,
%$\Lambda \, C_0 = -2 \, x_1$,
%$\Lambda \, C_1 = -2 \, x_1^2 = -2 \, R + 2  \, x_1$,
%and
\[
  \Lambda
  \left( \begin{array}{c}
  C_0 \\
  C_1
  \end{array} \right)
  =
  \left( \begin{array}{cc}
  0     & -2 \\
  -2R   &  2
  \end{array}\right)
  \left( \begin{array}{c}
  C_0 \\
  C_1
  \end{array} \right).
\]
With $\Lambda \rightarrow \lambda$, the characteristic equation (\ref{eq:xr}) reads
\begin{equation}
0=  \left| \begin{array}{cc}
  \lambda     & 2          \\
  2 R         & \lambda - 2
  \end{array} \right| = -4R - 2\lambda + \lambda^2.
  \tag{\ref{eq:xr}-1}
\label{eq:xr1}
\end{equation}
Thus, $R = (\lambda^2 -2\lambda)/4$;
  and the fixed point begins at $R_a = -1/4$ ($\lambda = +1$),
  and becomes unstable at $R_b = 3/4$ ($\lambda = -1$).



For $n = 2$, the cyclic polynomials
  are
$C_0 = 1$,
$C_1 = x_1 + x_2$,
$C_{12} = x_1 x_2$;
and $\Lambda = 4 \, x_1 x_2$.
Thus,
$\Lambda \, C_0 = 4 \, C_{12}$,
$\Lambda \, C_1 = 4 \, {x_1}^2 \, x_2 + 4 \, x_1 \, {x_2}^2
  = -8 \, R \, C_0 + 4 \, (R + 1) \, C_1$
(see the example after \refthm{sqrfree}),
$\Lambda \, C_{12}
  = 4 \, (R - x_2) (R - x_1)$,
and
\[
\Lambda
  \left( \begin{array}{c}
  C_0 \\
  C_1 \\
  C_{12}
  \end{array} \right)
 =
  \left( \begin{array}{ccc}
  0           & 0         & 4 \\
  -8R         & 4(R+1)    & 0 \\
  4R^2        & -4R       & 4
  \end{array} \right)
  \left( \begin{array}{c}
  C_0 \\
  C_1 \\
  C_{12}
  \end{array} \right).
\]
%
%
%
The characteristic equation \refeq{xr} reads
%
%
%
\begin{align}
0 &= (4R - 4 + \lambda) [(4 R-\lambda)^2 - 4 \lambda].
  \tag{\ref{eq:xr}-2}
\label{eq:xr2}
\end{align}
%
%
%
The first factor is responsible for the 2-cycle,
  while the second factor is for unstable fixed points. % (see later).
Setting the former to zero yields $R = 1 - \lambda/4$;
so the onset ($\lambda = +1$) value is $R_a = 3/4$ ($r_a = 3$)
and the bifurcation ($\lambda = -1$) value is $R_b = 5/4$ ($r_b = 1+\sqrt 6$).
%
%Note that the onset point of the 2-cycle
%  is located at $R = 3/4$,
%  where the fixed point bifurcates \cite{strogatz}.
%  \big[compare \reffigs{cobweb}(a) and (b)\big].




The $n = 3$ case is nontrivial and has sparked several different
  elementary derivations \cite{brown1, saha, bechhoefer, gordon, burm, zhang}.
For the current algorithm,
%the cyclic polynomials are
%$C_0 = 1$,
%$C_1 = x_1 + x_2 + x_3$,
%$C_{12} = x_1 x_2 + x_2 x_3 + x_3 x_1$,
%$C_{123} = x_1 x_2 x_3$,
%and
$\Lambda = -8 \, C_{123}$,
%
and the square-free reduction yields
%
\begin{small}
\begin{align*}
\Lambda
\left( \begin{array}{c}
  C_0 \\
  C_1 \\
  C_{12} \\
  C_{123}
\end{array} \right)
=
\left( \begin{array}{cccc}
 0            & 0                 & 0               & -8 \\
 -24 R        & 8(R+1)            & -8R             & 0 \\
 24 R^2       & -8R(R+2)          & 8(R+1)          & 0 \\
 -8R^3        & 8R^2              & -8R             & 8
  \end{array} \right)
\left( \begin{array}{c}
  C_0 \\
  C_1 \\
  C_{12} \\
  C_{123}
\end{array} \right),
\end{align*}
\end{small}
%
%
%
and \refeq{xr} reads:
\begin{align}
  0
&= -\big[
  64R^3 - 128 R^2 - 8 (\lambda - 8) R - (\lambda -8)^2
  \big] \notag \\
&\qquad
  \big(
  \lambda^2 - 8\lambda - 24 R\lambda - 64 R^3
  \big).
  \tag{\ref{eq:xr}-3}
\label{eq:xr3}
\end{align}
%
Using the first factor, we find at the onset point
  $\lambda = 1$,
  $\left(
  R-\frac74
  \right)
  \left(
  R^2-\frac14 R + \frac{7}{16}
  \right)=0$
  and its only real solution is $R_a = 7/4$ ($r_a = 1+\sqrt 8$).
%
At the bifurcation point $\lambda = -1$,
  the equation $R^3-2R^2 + \frac{9}{8} R -\frac{81}{64}=0$
  yields
  $R_b = \frac14
      \left(
      \frac{8}{3} +
        \sqrt[3]{\frac{1915}{54} - \frac52\sqrt{201}}
       +\sqrt[3]{\frac{1915}{54} + \frac52\sqrt{201}}
      \right)$,
whose corresponding $r = 1+\sqrt{1+4R}$
is identical to that in ref. \cite{gordon, burm}.
%




For the $n = 4$ case
\cite{brown1, stephenson1, stephenson3, bailey, kk1, lewis},
%$C_0 = 1$,
%$C_1 = x_1 + \cdots + x_4$,
%$C_{12} = x_1 x_2 + \cdots + x_4 x_1$,
%$C_{13} = x_1 x_3 + x_2 x_4$,
%$C_{123} = x_1 x_2 x_3 + \cdots + x_4 x_1 x_2$,
%$C_{1234} = x_1 x_2 x_3 x_4$,
%and
%$\Lambda = 16 \, C_{1234}$.
%
we only list the final characteristic equation \eqref{eq:xr}
after the square reduction (with six square-free cyclic polynomials):
%
\begin{small}
\begin{align}
0
%& = \left|
%\begin{array}{cccccc}
% \lambda  & 0               & 0                 & 0     & 0     & -16 \\
% 64 R     & \lambda-16(R+1) & 16 R              & 0     & -16R  & 0 \\
% -64 R^2  & 16R(R+2)        & \lambda-16(R^2+1) & -32 R & 16R   & 0 \\
% -32 R^2  & 16R(R+1)        & -16R              & \lambda-16(R^2+1)  & 0 & 0 \\
% 64 R^3   & -16R^2(R+3)     & 16R(R+2)        & 32 R (R+1)       & \lambda-16(R+1) & 0 \\
% -16R^4   & 16R^3           & - 16R^2           & -16R^2            & 16R & \lambda - 16
%\end{array}
%\right | \nonumber  \\
& =
    \big[
    4096 R^6-12288 R^5 + 256 (\lambda + 48) (R^4- R^3) \notag \\
& \quad
    -16 (\lambda + 32)(\lambda - 16) R^2
      -(\lambda - 16)^3 \big] \notag \\
& \quad
    \big[ 16 (R-1)^2 - \lambda \big]
 \, \big[(16 R^2 + \lambda)^2 - 16 (2R+1)^2 \lambda \big],
  \tag{\ref{eq:xr}-4}
\label{eq:xr4}
\end{align}
\end{small}
%
%
%
The first (sextic) factor, responsible for the 4-cycle,
can be factorized as
$ (
    4R - 5
  )
\, \big[
  (4R + 1)^2 + 4
  \big]
\, \big[
  (4R - 3)^3 - 108
  \big] =0$
at the onset point $\lambda = 1$.
%
%
%
It has two real roots:
$R_a' = 5/4$
\big($r_a' = 1+\sqrt{6} \approx 3.4495$\big)
for the cycle from period-doubling the 2-cycle
  \big[compare \reffigs{cobweb}(b) and (d)\big],
and
$R_a = (3+\sqrt[3]{108})/4$
\big($r_a = 1+\sqrt{4+\sqrt[3]{108}} \approx 3.9601$\big)
for an original cycle \big[\reffig{cobweb}(e)\big].
%
At the bifurcation point ($\lambda = -1$), we have
$4096 R^6 - 12288 R^5 + 12032 (R^4 - R^3)
  + 8432 R^2 + 4913 = 0$,
which upon $R \rightarrow r(r-2)/4$ yields the same polynomial
obtained previously
\cite{bailey, kk1, lewis}.
%
The only two positive roots $r_b \approx 3.9608$
and $r_b' \approx 3.5441$ of the sextic equation
correspond to $r_a$ and $r_a'$ respectively.
%
%Since the 4-cycle problem were tackled mainly on computers
%  in the previous studies,
%  we also supply a compact elementary derivation
%  in the Appendix.




The algorithm was coded into a Mathematica program,
which was used to compute the polynomials for $n$ up to 13.
%
For polynomials of larger $n$ are saved in in the website in \refsec{end}.
%
The representative $r$ values are listed in \reftab{rval}.
%




\begin{table}[h]\footnotesize
  \tbl{
  Smallest positive $r$ at the onset and bifurcation points
  of the $n$-cycles of the logistic map.
  }
{
\begin{tabular}{llll}
%\begin{tabularx}{\linewidth}{
%  >{\hsize=0.5\hsize\centering\arraybackslash}X
%  >{\hsize=1.6\hsize}X
%  >{\hsize=1.6\hsize}X
%  >{\hsize=0.3\hsize\raggedright\arraybackslash}X
%}
\hline
  $n^\dagger$
& Onset$^\ddagger$
& Bifurcation$^\ddagger$
& \#$^*$ \\
\hline
$1$     & $1.0000000000_1$      &  $3.0000000000_1$       & 1   \\
$2'$    & $3.0000000000_1$      &  $3.4494897428_1$       & 1   \\
$3$     & $3.8284271247_1$      &  $3.8414990075_3$       & 1   \\
$4$     & $3.9601018827_3$      &  $3.9607686524_6$       & 1   \\
$4''$   & $3.4494897428_1$      &  $3.5440903596_6$       & 1   \\
$5$     & $3.7381723753_{11}$   &  $3.7411207566_{15}$    & 3   \\
$6$     & $3.6265531617_{20}$   &  $3.6303887000_{27}$    & 4   \\
$6'$    & $3.8414990075_{3}$    &  $3.8476106612_{27}$    & 1   \\
$7$     & $3.7016407642_{57}$   &  $3.7021549282_{63}$    & 9   \\
$8$     & $3.6621089132_{108}$  &  $3.6624407072_{120}$   & 14  \\
$8'$    & $3.9607686524_{6}$    &  $3.9610986335_{120}$   & 1   \\
$8'''$  & $3.5440903596_{6}$    &  $3.5644072661_{120}$   & 1   \\
$9$     & $3.6871968733_{240}$  &  $3.6872742105_{252}$   & 28  \\
$10$    & $3.6052080669_{472}$  &  $3.6059169323_{495}$   & 48  \\
$10'$   & $3.7411207566_{15}$   &  $3.7425706462_{495}$   & 3   \\
$11$    & $3.6817160194_{1013}$ &  $3.6817266457_{1023}$  & 93  \\
$12$    & $3.5820230011_{1959}$ &  $3.5828117795_{2010}$  & 165 \\
$12'$   & $3.6303887000_{27}$   &  $3.6321857392_{2010}$  & 4   \\
$12''$  & $3.8476106612_{27}$   &  $3.8490363152_{2010}$  & 1   \\
$13$    & $3.6797024578_{4083}$ &  $3.6797038498_{4095}$  & 315 \\
\hline
\multicolumn{4}{p{\linewidth}}{
$^\dagger$
  $\,'$, $\,''$, or $\,'''$ means
    that the cycle is undergoing
    the first, second, or third successive period-doubling, respectively.
} \\
\multicolumn{4}{p{\linewidth}}{
$^\ddagger$
  The subscripts are the degrees of the corresponding minimal polynomial
    of $R = r(r-2)/4$.
} \\
\multicolumn{4}{p{\linewidth}}{
$^*$
  The number of similar cycles.
} \\
\hline
%\end{tabularx}
\end{tabular}
\label{tab:rval}
}
\end{table}







\subsection{\label{sec:primfac}Minimal polynomial of the $n$-cycles}


The characteristic polynomial $A_n\Rlam$ in \refeq{xr}
  is derived for the fixed points of $f^n$.
Thus, it represents not only the $n$-cycles,
  but also cycles of shorter periods $d$ ($d|n$).
%
We can remove the contributions from the shorter cycles by the following theorem.


\begin{theorem}
  The minimal polynomial $P_n\Rlam$ of all $n$-cycles
  is a factor of the characteristic polynomial $A_n\Rlam$,
  and can be computed as
  \begin{equation}
    P_n\Rlam
    = \prod_{cd = n} B_{d,c}\Rlam^{\mu(c)},
  \label{eq:primfac}
  \end{equation}
where
  $B_{d, c}\Rlam \equiv \prod_{k=1}^c A_{d}(R, e^{2k\pi i/c} \lambda^{1/c})$,
  $\lambda^{1/c}$ is a complex $c$th root of $\lambda$,
  and $\mu(c)$ is the M\"obius function.
  \label{thm:primfac}
\end{theorem}
%
%
%\begin{remark}[1]
\begin{remark}
The M\"obius function $\mu(n)$ is $(-1)^k$
  if $n$ is the product of $k$ distinct primes,
  or 0 if $n$ is divisible by a square of a prime.
%Thus,
$\mu(n)$ = 1, $-1$, $-1$, 0, $-1$, 1, \ldots,
  from $n = 1$%.
%The $\mu(n)$ is useful for inversion:
%$g(n) = \sum_{d|n} \mu(n/d) \, h(d)$
%if and only if $h(n) = \sum_{d|n} g(d)$
\cite{hardy}.
\end{remark}
%
%
%\begin{remark}[2]
\begin{remark}
%$B_{d,c}\Rlam$ is a polynomial of $\lambda$.
Despite the argument $\lambda^{1/c}$,
  the product $\prod_{k=1}^c A_d(R, e^{2k\pi i/c} \lambda^{1/c})$,
  hence $B_{d,c}\Rlam$,
  is free from radicals of powers of $\lambda^{1/c}$,
  for it is invariant under $\lambda \rightarrow e^{2\pi i} \lambda$.
Thus, $B_{d, c}\Rlam$ is a polynomial of $\lambda$,
and $\deg_\lambda B_{d,c}\Rlam = \deg_\lambda A_d\Rlam$.
Particularly, $B_{n,1}\Rlam = A_n\Rlam$.
\end{remark}
%
%
%We will refer to $P_n\Rlam$ as the \emph{primitive factor} below.
%
%
Let us apply \refeq{primfac}
to some examples in \refsec{examples}.
We will omit the argument $R$ below for convenience.
%%
For $n = 1$,
%there is no irrelevant factor in \refeq{xr1} and
$P_1(\lambda) = B_{1,1}(\lambda) = A_1(\lambda)
  = \lambda^2 - 2\,\lambda - 4\,R$.
%
%
%
For $n = 2$, since
$B_{1,2}(\lambda)
=
A_1(\sqrt{\lambda})
A_1(-\sqrt{\lambda})
=
(4R-\lambda)^2 -4\lambda$,
%
$P_2(\lambda) = A_2(\lambda) \, B_{1,2}(\lambda)^{-1} = 4R - 4 + \lambda$.
%
%
For $n = 4$,
%we have
%$16 (R-1)^2 - \lambda
%= (4R - 4 + \sqrt{\lambda})(4R - 4 - \sqrt{\lambda})$
%and
%$
%(16 R^2 + \lambda)^2 - 16 (2R+1)^2 \lambda
%=
%\big[(4R - \sqrt{\lambda})^2  - 4\sqrt{\lambda}\,\big]
%\big[(4R + \sqrt{\lambda})^2  + 4\sqrt{\lambda}\,\big]$.
%Thus,
we can show that
the last two factors of \refeq{xr4}
can be written as
$B_{2,2}(\lambda) = A_2(\sqrt{\lambda}) A_2(-\sqrt{\lambda})$,
and
%
$P_4(\lambda)
    = A_4(\lambda) \, B_{2,2}(\lambda)^{-1}$
indeed the yields the sextic factor.
%$    = 4096 R^6-12288 R^5+ 256 (\lambda + 48) (R^4- R^3)
%    -16(\lambda + 32)(\lambda - 16) R^2 - (\lambda-16)^3$.
Note, $B_{1,4}(\lambda)$ is unused for $\mu(4) = 0$.






For $n \ge 4$,
  $P_n\Rlam$ can be easily recognized,
  without using \refthm{primfac},
  as the factor of $A_n\Rlam$
  with the highest degree in $R$.
  %see \reftab{Anlog} and \refsec{degR}.
It can be, however, problematic,
   if $P_n(R, \lambda)$ is solved for $\lambda = 1$
   instead of a general $\lambda$,
   for $P_n(R, 1)$ itself can be further factorized.
%
Due to the technical nature of the derivation and subsequent discussions,
  the reader may wish to skip the rest of \refsec{logmap}
  on first reading.






\subsection{\label{sec:degprimfac}Counting cycles}



%We will prove \refthm{primfac} by the following steps.
%
To show \refthm{primfac}, we first find the degrees in $\lambda$ of
  $A_n\Rlam$ (\refthm{necklace}) and $P_n\Rlam$ (\refthm{lyndon}).
By comparing the degrees,
  we then show that each $P_d\Rlam$ with ($d|n$),
  after some transformation,
  contributes one polynomial factor to $A_n\Rlam$
  (\refthm{prod}),
  and the inversion of the relation
  yields \refthm{primfac}.



\subsubsection{Number of the square-free cyclic polynomials}

To count the square-free cyclic polynomials,
  we establish a one-to-one mapping
  between the square-free cyclic polynomials
  and the binary necklaces.
%
A binary necklace
  is a nonequivalent binary $0$-$1$ string.
%
Two strings are equivalent if they differ only by a circular shift.
%
For example, for $n = 3$, there are $2^3 = 8$ binary strings,
but only four necklaces: 000, 001, 011 and 111,
since 010 and 100 are equivalent to 001,
so are 110 and 101 to 011.
%
If the generator of the cyclic polynomial contains $x_k$,
  the $k$th character from the left of the necklace is 1,
  otherwise 0,
e.g. the above four necklaces correspond to
  $C_0$, $C_1$, $C_{12}$ and $C_{123}$, respectively.
%For example, if $n = 3$,
% the square-free polynomials for $000$, $100$, $110$ and $111$ are
%  $C_0 = 1$ (generator: 1),
%  $C_1 = x_1 + x_2 + x_3$ (generator: $x_1$),
%  $C_{12} = x_1 x_2 + x_2 x_3 + x_3 x_1$ (generator: $x_1 x_2$)
%  and $C_{123} = x_1 x_2 x_3$ (generator: $x_1 x_2 x_3$), respectively.
Thus,

\begin{theorem}
The number $\NB(n)$ of the square-free cyclic polynomials $C_p(\vx)$
  formed by the cycle points $\vx = \{x_1, \ldots, x_n\}$,
  which is also $\deg_\lambda A_n\Rlam$, is given by
\begin{equation}
  %\deg_\lambda A_n\Rlam =
  \NB(n) = N(n) = (1/n) \textstyle\sum_{d|n} \varphi(n/d) 2^{\,d}.
\label{eq:necklace}
\end{equation}
%
\label{thm:necklace}
\end{theorem}
%
Here, $\varphi(m)$ is Euler's totient function,
which gives the number of integers from 1 to $m$
  that are coprime to $m$,
  e.g.,
  $\varphi(1) = 1$,
  $\varphi(3) = 2$ for 1 and 2,
  $\varphi(6) = 2$ for 1 and 5.
The first few values of $N_\Pset(n)$ are 2, 3, 4, 6, 8, 14, 20, 36, 60, 108, 188, 352, 632, \ldots,
  from $n = 1$.



\subsubsection{Number of the $n$-cycles}


\begin{theorem}
The degree in $\lambda$ of the minimal polynomial $P_n\Rlam$ of all $n$-cycles
  is equal to the number of the $n$-cycles,
  which is given by
\begin{equation}
  %\deg_\lambda P_n\Rlam =
  L(n) = (1/n)\textstyle\sum_{d | n} \mu(n/d) 2^{\,d}.
\label{eq:lyndon}
\end{equation}
\label{thm:lyndon}
\end{theorem}
%
%
\begin{proof}
Let $\lambda_c$ be the value of $\Lambda(\vx^c)$
  evaluated at the $c$th cycle.
The minimal polynomial $P_n\Rlam$
  can be written as $\prod_c (\lambda - \lambda_c)$,
  and $\deg_\lambda P_n\Rlam$ is the same as
  the number $L(n)$ of $n$-cycles, which is known \cite{hao, hao2, lutzky}.
\end{proof}
%
%
The first few values of $L(n)$ are
2, 1, 2, 3, 6, 9, 18, 30, 56, 99, 186, 335, 630, \ldots,
from $n = 1$.
%
Incidentally, $L(n)$ is also the number of aperiodic necklaces of length $n$.
  In the $n = 4$ case,
  $0001$, $0011$, and $0111$ are aperiodic;
  but $0000$, $1111$, and $0101$ are not
  owing to the shorter periods 1, 2, and 2, respectively.
%
Since an $n$-necklace is either aperiodic itself or
  a repeated aperiodic $d$-necklace (with $d|n$),
  we have
\begin{equation}
  N(n) = \textstyle\sum_{d|n} L(d).
\label{eq:necklacelyndon}
\end{equation}
%



\subsubsection{Relation between $P_n$ and $A_n$}



\begin{theorem}
The minimal polynomials $P_d\Rlam$ of all $d$-cycles
  of periods $d|n$
and $A_n\Rlam$ [defined in \refeq{xr}] are related by
  \begin{equation}
    A_n\Rlam = \prod_{cd = n} Q_{d, c}\Rlam,
    \label{eq:prod}
  \end{equation}
  where
  $Q_{d,c}\Rlam = \prod_{k=1}^{c} P_d(R, e^{2 k \pi i/c} \lambda^{1/c})$
  is a polynomial of degree $L(d)$ in $\lambda$
  representing contributions from $d$-cycles.
%encompassed by $A_n\Rlam$.
\label{thm:prod}
\end{theorem}

\begin{proof}
Since $A_n\Rlam$ represent all $d$-cycles with $d|n$,
and each cycle holds a distinct $\lambda$,
  $\deg_\lambda A_n\Rlam$
  is at least $\sum_{d|n} L(d)$
  according to \refeq{lyndon},
  which is equal to
  $N(n)$ by \refeq{necklacelyndon}.
Since $\deg_\lambda A_n\Rlam = N(n)$ by \refeq{necklace},
  each $n$-cycle occurs exactly once in $A_n\Rlam$.


In a $d$-cycle,
 we have $\Lambda(\{x_1, \ldots, x_n\}) = (-2)^n x_1 \dots x_n
   = {\Lambda_d}^c$,
where
$\Lambda_d(\{x_1, \ldots, x_d\}) \equiv (-2)^d x_1 \dots x_d$
and
$c \equiv n/d$.
So the $d$-cycle satisfies a polynomial
  $P_d(R, \lambda^{1/c}) = 0$,
where $\lambda = \Lambda(\{x_1,\ldots,x_n\})$.
%
This is, however, not a polynomial equation, and
%
the radical $\lambda^{1/c}$ can be removed by the product
$Q_{d,c}\Rlam
  \equiv \prod_{k=1}^c P_d(R, e^{2k\pi i/c} \, \lambda^{1/c}) = 0$.
%
Now $Q_{d,c}\Rlam$ is a polynomial of $\lambda$
  for it is invariant under
  $\lambda \rightarrow e^{2\pi i} \lambda$,
and thus free from radicals of the form $\lambda^{l/c}$
  \big[if $(l, c) \ne c$\big].
And since
  $\deg_\lambda Q_{d,c}\Rlam
    = \deg_\lambda P_d\Rlam = L(d)$,
  it is also a polynomial
  of the lowest possible degree in $\lambda$.


Therefore, the product $\prod_{cd=n} Q_{d,c}\Rlam$
  can differ from $A_n\Rlam$ only by
  a multiple.
Since $Q_{n,1}\Rlam = P_n\Rlam$,
  and the coefficient of highest power of $\lambda$ is always
  unity in $A_n\Rlam$ \big[see the definition \refeq{xr}\big],
  we know by induction that the coefficients of the highest power of $\lambda$
  in all $P_n\Rlam$ and $Q_{d,c}\Rlam$
  are also unities.
So the multiple is one, hence \refeq{prod}.
%
%
%
\end{proof}




\refthm{primfac} is a corollary of \refthm{prod},
\begin{small}
\begin{align*}
 B_{n, m} (\lambda)
% & = \prod_{l = 1}^{m} A_n(e^{2l\pi i/m} \lambda^{1/m}) \\
 & = \prod_{c | n} \prod_{l=1}^m \prod_{k=1}^c
    P_{n/c}(e^{\frac{2k\pi i}{c} + \frac{2l\pi i}{m c}} \lambda^{1/(m c)}) \\
 & = \prod_{c | n} \prod_{k'=1}^{m c}
    P_{n/c}(e^{2k'\pi i/(m c)} \lambda^{1/(m c)}) \\
 & = \prod_{c | n} Q_{n /c, m c}(\lambda).
\end{align*}
\end{small}
The M\"obious inversion yields
\begin{equation}
  Q_{n, m}\Rlam
= \prod_{d|n} B_{d, m n/d}\Rlam^{\mu(n/d)},
\tag{$\ref{eq:primfac}'$}
\end{equation}
which is reduced to \refeq{primfac} with $m = 1$,
for $Q_{n, 1}\Rlam = P_n\Rlam$.









\subsection{\label{sec:origfac}%Intersection of cycles and
  Factors at the onset point}


The onset polynomial
  $P_n(R, \lambda = 1)$ for the $n$-cycles
  can be further factorized.
%
This is caused by the intersection of an $n$-cycle
  and a shorter $d$-cycle ($d|n$, $d < n$),
  which forces the two to share orbit.
%
One can show that this intersection happens only
  at the onset of the $n$-cycle, where $\lambda = 1$,
%
  and the $d$-cycle branches out, or ``bifurcates'', $n/d$-fold
  to become the $n$-cycle.
%
The simplest example is the first bifurcation point
  at $R = 3/4$ for $d = 1, n = 2$,
  where the fixed point \refeq{xr1}
  undergoes period-doubling and becomes the 2-cycle \refeq{xr2}.
As a result, the polynomial $P_n(R, \lambda)$ from the $n$-cycle
  has to accommodate $P_d(R, \lambda')$ from the $d$-cycle as a factor,
  with $\lambda'$ being a primitive $(n/d)\,$th root of $\lambda = 1$.



If an $n$-cycle is not born out of the above branching,
  we call it \emph{original}.
The two types of cycles exist
  as separate factors of $P_n(R, \lambda = 1)$.
One can then show that the factor $S_n(R)$ responsible for the original cycles
  %or the \emph{original factor} below,
  is given by
  \begin{equation}
    S_n(R)
    = \frac
    {
      P_n(R, 1)
    }
    {
      \prod_{c d =  n, \; c > 1}
      %\left[ \,
        \prod_{(k, c) = 1}
      P_{d}
        \left(
          R, e^{2k\pi i/c}
        \right)
      %\right]
    },
  \label{eq:origfac}
  \end{equation}
where the inner product on the denominator is carried over
   $k$ from 1 to $c$ that are coprime to $c$.



Take the $n = 4$ case as an example, we have
$\prod_{k=1,3} P_1(R, e^{k\pi i/2})
%=(-4R-2i-1)(-4R+2i-1)
=16R^2+8R+5$
(for $c = 4$),
$P_2(R, -1) = 4R - 5$
(for $c = 2$),
and
$P_4(R, 1)
= (4R - 5) (16 R^2 + 8 R + 5)
  \bigl[
    (4R - 3)^3 - 108
  \bigr]$.
Dividing $P_4(R, 1)$ by the two factors yields
$S_4(R) = (4R-3)^3 - 108$,
  whose only real root $R_a=(3+\sqrt[3]{108})/4$ corresponds to
  the onset of the original cycle.
Note $R_a' = 5/4$ is excluded from $S_4(R)$
  as it is due to period-doubling the 2-cycle.
The two cycles are compared in \reffigs{cobweb}(d) and (e).



%We now prove \refthm{origfac}.
%Suppose $n = c \,d$, we have, from \refeq{logmaps},
%\[
%  x_{l+1} - x_{d+l+1} = - (x_l + x_{d+l}) (x_l - x_{d+l}).
%\]
%We apply the equation to $l = 1, \ldots, m$, and the product is
%\[
%  x_{m+1} - x_{m + d+1} =
%  (-1)^{m} \left[ \, \prod_{l=1}^{m} (x_l + x_{d+l}) \right]
%    (x_1 - x_{d+1}).
%\]
%We now set $m$ to $0, d, \ldots, (c-1)\,d$ in this equation,
%  add them together, eliminate $x_1 - x_{d+1}$
%  (which is nonzero in a cycle), and
%\begin{equation}
%  \sum_{c' = 0}^{c-1}
%    (-1)^{c' d} \prod_{l=1}^{c' d} (x_l + x_{d+l})
%   = 0.
%\label{eq:stair}
%\end{equation}
%%
%Note \refeq{stair} holds for every divisor $c$ of $n$ ($c > 1$). We now have
%
%
%
%\begin{theorem}
%An $n$-cycle and a shorter $d$-cycle ($d|n$, $d< n$)
%  intersect only at the onset of the $n$-cycle,
%%
%and $\prod_{k=1}^{d} f'(x_k) = (-2)^d x_1 \dots x_d$ is
%a \emph{primitive} $(n/d)$th root of unity there.
%\label{thm:cbifur}
%\end{theorem}
%
%
%\begin{proof}
%At the intersection of the $n$- and $d$-cycles,
%  $x_l$ repeats itself after $d$ steps, so
%  $x_{d + l} = x_l$;
%and \refeq{stair} becomes,
%%
%\begin{equation}
%  1 + q + \dots + q^{c-1} = 0,
%\label{eq:cbifur}
%\end{equation}
%%
%where $q = (-2)^d \, x_1 \dots x_d$.
%%
%Multiplying \refeq{cbifur} by $q - 1$ yields
%  $1 = q^c = (-2)^n \, x_1 \dots x_n$.
%So the $n$-cycle is at its onset.
%
%
%Further $q$ is a \emph{primitive} $c$th root of unity.
%%
%Suppose the contrary: $q = e^{2k\pi i/c}$ and $(k, c) = g >1$,
%  then by $c_1 \equiv c/g$, $k_1 \equiv k/g$, we have
%\begin{equation}
%  q = e^{2k_1\pi i/ c_1}.
%\label{eq:qd}
%\end{equation}
%Similar to \refeq{cbifur}, we can apply \refeq{stair} with $c \rightarrow g$ and $d \rightarrow d c_1$, and
%\[
%  1 + q_1 + \dots + {q_1}^{g-1} = 0,
%\]
%  where $q_1 = (-2)^{d c_1} \, x_1 \dots x_{dc_1} = q^{c_1}$.
%But by \refeq{qd}, $q^{c_1} = e^{2 k_1\pi i} = 1$,
%and $1 + q_1 + \dots + {q_1}^{g-1} = g > 0$; a contradiction.
%%
%\end{proof}
%
%
%\begin{remark}
%The only real $q$ is $q = -1$ for $c=2$, i.e., a period-doubling.
%%
%On the complex domain, however, we can have a $c$-fold branching with $c>2$,
%  which corresponds to a contact points between ``bulbs''
%  in the inverted Mandelbrot set.
%\end{remark}
%
%By \refthm{cbifur}, $P_n\Rlam$ at the onset point
%includes $P_d(R, e^{2 k \pi i/c} \lambda^{1/c})$ for every
%  possible combination of $k$ and $c$, such that
%  $(k, c) = 1$, $c|n$, and $c > 1$.
%
%Dividing the factors from $P_n\Rlam$ yields \refthm{origfac}.








\subsection{\label{sec:degR}Degrees in $R$}



%\begin{theorem}
The degrees in $R$ of $A_n\Rlam$, $P_n\Rlam$ and $S_n(R)$ are
\begin{subequations}\small
\begin{align}
\deg_R A_n\Rlam &= \sum_{d|n} \varphi(n/d) 2^{d-1}, \\
\deg_R P_n\Rlam &\equiv \beta(n) = \sum_{d|n} \mu(n/d) 2^{d-1}, \\
\deg S_n(R) &= \beta(n) - \sum_{d | n, d < n} \beta(d) \, \varphi(n/d).
\end{align}
\label{eq:degR}
\end{subequations}
%\label{thm:degR}
%\end{theorem}
%
%
%
\begin{proof}
We first prove \refeqsub{degR}{a}.
%
We recall the subscript $p$ of $C_p$ denotes a sequence of indices $k$
  in the generating monomial $\prod_k {x_k}^{e_k}$.
But for a square-free cyclic polynomial,
  each $k$ occurs no more than once,
so $p$ also represents a \emph{set} of indices,
%
e.g.,
$p = 1$ represents $\{1\}$,
  ($C_1 = x_1 + \dots + x_n$, generator: $x_1$)
and
$p = 13$ represents $\{1, 3\}$
  ($C_{13} = x_1 \, x_3 + x_2 \, x_4 + \dots + x_n \, x_2$,
  generator: $x_1 \, x_3 $, assuming $n \ge 4$).
%more examples are listed in \reftab{sqrfreepoly}).
%
In this proof,
we shall also use $p$ to denote the corresponding index set,
  $|p|$ the set size, i.e., the number of indices in the set,
  and
  $\bar p \equiv \{1,\ldots,n\} \backslash p$ the complementary set.
%
Obviously, $|\bar p| + |p| = n$.
%
Further, we will include $p$ that correspond to alternative generators
  of the same cyclic polynomial,
e.g., we allow $p = \{2\}$, $\{3\}$, \dots, $\{n\}$,
  although they represent the same cyclic polynomial $C_1$
  as $p = \{1\}$.


Next, we recall the matrix elements $T_{pq}(R)$ arise from the square-free reduction
of $\Lambda(\vx) C_p(\vx) = (-2)^n x_1 \dots x_n \, C_p(\vx)$.
%
A single replacement ${x_k}^2 \rightarrow R - x_{k+1}$
produces two new terms:
in the first, ${x_k}^2 \rightarrow R$,
and in the second, ${x_k}^2 \rightarrow - x_{k+1}$.
We call the two type 1 and type 2 replacements, respectively.
%
If a monomial term $t(R, \vx)$
results from $l_1$ type 1 and $l_2$ type 2 replacements
during the reduction of a term $s(\vx)$ in $\Lambda(\vx) C_p(\vx)$,
then the degrees in $\vct x$, for any $x_k$, of $s(\vx)$ and $t(R, \vx)$
are related as
%
\begin{equation}
  \deg_\vx s(\vx) - \deg_\vx t(R, \vx) = 2 l_1 + l_2.
\label{eq:degxdiffst}
\end{equation}
%
Similarly, the degrees in $R$ satisfy
\[
  \deg_R s(\vx) - \deg_R t(R, \vx) = - l_1,
\]
but since $\deg_R s(\vx) = 0$,
\begin{equation}
  \deg_R t(R, \vx) = l_1.
\label{eq:degRt}
\end{equation}

%
Now if the monomial $t(R, \vx)$ settles in the $q$th column of
  the matrix $\vct T(R)$, as part
  of $T_{pq}(R) C_q(\vx)$ in \refeq{xcp0},
then
  $t(R, \vx)$
  must be
%the product of
%  a monomial of $R$ \big[as part of $T_{pq}(R)$\big]
%  and one of the
  a generator of $C_q(\vx)$;
so
%
\begin{equation}
  \deg_\vx t(R, \vx) = |q|.
\label{eq:degxt}
\end{equation}
%
Since $s(\vx)$ is part of $\Lambda(\vx) C_p(\vx)$,
  we have
%
\begin{equation}
  \deg_\vx s(\vx) = \deg_\vx \Lambda(\vx) + \deg_\vx C_p(\vx)
  = n + |p|.
\label{eq:degxs}
\end{equation}
%
%
From \refeqs{degxdiffst}, \req{degxt}, \req{degxs}, we get
\[
  n + |p| - |q| = 2 \, l_1 + l_2,
\]
and
\begin{equation}
  l_1  =    (n + |p| - |q| - l_2)/2
             \le  (n + |p| - |q|)/2.
\label{eq:l1limit}
\end{equation}
By \refeq{degRt}, we get
\begin{align*}
  \deg_R T_{pq}(R)
  = \max\{ \deg_R t(R, \vx) \}
  = \max\{ l_1 \}
  \le   (n + |p| - |q|)/2,
\end{align*}
where the equality holds when all replacements are type 1 ($l_2 = 0$).


Finally, each term of the determinant
$A_n\Rlam = \big|\lambda \, \vct I - \vct T(R)\big|$
is given by
$(-1)^s \prod_{p} \big[\lambda \, \delta_{p q} - T_{pq}(R)\big]$,
where $p$ runs through rows of the matrix
  and $\{q\}$ is a permutation of $\{p\}$,
  with $(-1)^s$ being the proper sign.
%
Summing over rows under this condition yields
\begin{align*}
  \deg_R A_n\Rlam
  = \max \sum_{p \in \Pset} \deg T_{pq}(R)
  \le N(n) \, \big(n + |p| - |q|\big)/2 = n N(n)/2,
\end{align*}
where equality can be achieved if $q = \bar p$ in every row.
By \refeq{necklace} we have \refeqsub{degR}{a}.



\refeqsub{degR}{b} and (c) follow from taking taking the degree in $R$ of
  \refeq{prod} and \refeq{origfac}.
\refeqsub{degR}{b} and (\ref{degR}c) were known \cite{mira, stephenson1, blackhurst}.
%
\end{proof}
%
The first few values of $\deg_R A_n\Rlam$,
  $\deg_R P_n\Rlam$,
  and $\deg S_n(R)$ are
  1, 3, 6, 12, 20, 42, 70, 144, 270, 540, 1034, 2112, 4108, \dots,
  1, 1, 3, 6, 15, 27, 63, 120, 252, 495, 1023, 2010, 4095, \dots,
  1, 0, 1, 3, 11, 20, 57, 108, 240, 472, 1013, 1959, 4083, \dots,
  respectively (starting from $n = 1$).




\section{\label{sec:henon}H\'enon map}




A simple two-dimensional extension of the logistic map
  is the H\'enon map \cite{henon}:
\begin{equation}
  x_{k+1} = 1 + y_k - a \, {x_k}^2, \quad
  y_{k+1} = b \, x_k.
\label{eq:henon}
\end{equation}
%
By a change of variables
  $x_k^\mathrm{new} \equiv a \, x_k^\mathrm{old},
   y_k^\mathrm{new} \equiv a \, y_k^\mathrm{old}$,
we have, in terms of the new variables,
\begin{equation}
  x_{k+1} = a + y_k - {x_k}^2, \quad
  y_{k+1} = b \, x_k.
\tag{$\ref{eq:henon}'$}
\label{eq:henons}
\end{equation}
%
%Since neither $a$ nor $b$ is changed during the transformation,
\refeq{henon} and \refeq{henons} share the
same onset and bifurcation points
in terms of $a$ and $b$.
%
With $b \rightarrow 0$ and $a \rightarrow R$,
\refeq{henons} is reduced to the logistic map \refeq{logmaps}.


The second equation of \refeq{henons}
allows us to replace $y_k$ by $b \, x_{k-1}$,
  such that we can work with the cyclic polynomials of $x_k$ only.
The square-free reduction is now
  ${x_k}^2 \rightarrow a + b \, x_{k-1} - x_{k+1}$.



The counterpart of \refeq{der}
  can be found from the Jacobian of $n$th iterate of \refeq{henons}
\[
  \vct J^n
  \equiv
  \left(
    \begin{array}{ccc}
      \partial x_{n+1}/\partial x_1 & \partial x_{n+1}/\partial y_1 \\
      \partial y_{n+1}/\partial x_1 & \partial y_{n+1}/\partial y_1 \\
    \end{array}
  \right),
\]
and the magnitude of both eigenvalues of the matrix must be less than 1.
By the chain rule, we have
  $\vct J^n (x_1) = \vct J^1(x_1) \cdots \vct J^1(x_n)$,
  where
$\vct J^1(x_k)
  =
  \left(
    \begin{array}{cc}
      -2 x_k & 1 \\
      b & 0
    \end{array}
  \right)$.
Thus, the eigenvalues $\lambda$ of $\vct J^n$ can be found from
%
\begin{equation}
\big|\, \lambda \, \vct I - \vct J^n(x_1) \,\big|
  = \lambda^2 - \Theta(\vx) \,\lambda +(-b)^n = 0,
%\tag{$\ref{eq:der}'$}
\label{eq:henonder}
\end{equation}
%
where $\Theta(\vx)$ and $(-b)^n$
  are the trace and determinant of $\vct J^n(x_1)$,
  respectively \cite{hitzl}.
The onset and bifurcation points
  correspond to $\lambda = +1$ and $-1$, respectively



Using the matrix identity $\Tr( \vct A \vct B) = \Tr(\vct B \vct A)$,
  we have
    $\Tr \, \vct J^n(x_1) = \cdots = \Tr \, \vct J^n(x_n)$,
i.e., $\Theta(\vx)$ is a cyclic polynomial of $\vx = \{ x_k \}$.
%e.g., %$\Theta(\vx) = 4 x_1 x_2 + 2 \, b$ for $n = 2$ and
%$\Theta(\vx) = -8 x_1 x_2 x_3 - 2 \, b (x_1+ x_2+ x_3)$ for $n = 3$.
%
Thus, we can expand $\Theta(\vx) \, C_p(\vx)$
  in terms of square-free cyclic polynomials,
  and then replace $\Theta(\vx)$ by $\lambda + (-b)^n/\lambda$
  in the final characteristic equation to obtain the desired polynomials
  (the procedure is similar to that in \refsec{algo} for the logistic map).
%
%
%
The polynomials of $a$ and $b$
  at the onset and bifurcation points for $n$ up to 9
  are listed in the website in \refsec{end}.





\section{\label{sec:cubic}Cubic map}


For higher-order one-dimensional maps, we consider the following cubic map \cite{strogatz}
%
\begin{equation}
  x_{k + 1} = f(x_k) = r \, x_k - {x_k}^3.
\label{eq:cubic}
\end{equation}
This map is closely related to the one studied by Brown \cite{brown3, brown4}.
%
Unlike the logistic map, this map has a clear parity,
  that is, $f(-x) = -f(x)$.
We shall show below that this odd parity can be used to reduce the amount of calculation.

First of all, the new replacement rule
\begin{equation}
  {x_k}^3 \rightarrow r \, x_k - x_{k+1}
\label{eq:cubreplace}
\end{equation}
no longer eliminates squares.
%
We must, therefore, extend the basis set of cyclic polynomials
  from the square-free ones to the cube-free ones,
  in listing equations \eqref{eq:xcp}.
For example, we include in the basis
  $C_{112} = {x_1}^2 x_2 + {x_2}^2 x_3 + \dots + {x_n}^2 x_1$ ($n\ge3$),
but not
  $C_{1112} = {x_1}^3 x_2 + {x_2}^3 x_3 + \dots + {x_n}^3 x_1$.


The odd parity, however, allows us to limit
  the set of cube-free cyclic polynomials
  to those of even degrees in $\vx$.
%
This is because \refeq{cubreplace} contains only linear and cubic terms,
and a cyclic polynomial with an odd (even) degree in $\vx$
can never be reduced to one with an even (odd) degree.
%
%Thus, we only need polynomials of even degrees.
%
%We therefore have a theorem similar to \refthm{sqrfree}.
%%
%%
%\begin{theorem}
%  For the cubic map \refeq{cubic},
%  any cyclic polynomial $K(\vx)$ of
%  an $n$-cycle orbit
%  $\vct x = \{x_1, \ldots, x_n\}$
%  with an even degree in $\vct x$
%  is a linear combination of
%the even cube-free cyclic polynomials $C_p(\vx)$:
%\[
%  K(\vx) = \sum_{p \in \Pset} f_p(R) C_p(\vx),
%\]
%  where $\Pset = \{0, 11, 12, 13, \ldots, 1122, 1123, \ldots \}$ is
%  the set of indices of all even cube-free cyclic polynomials,
%  and $f_p(R)$ are polynomials of $R$.
%  \label{thm:cubfree}
%\end{theorem}
%
%
%
%
With the above change, the rest derivation is similar to that of the logistic map.
The new $\Lambda(\vx)$ should be
$\prod_{k=1}^n f'(x_k) = \prod_{k=1}^n (r - 3 {x_k}^2)$.
%
The polynomials of $r$
  at the onset and bifurcation points
%for some small $n$ are shown in \reftab{cubpolygen} (general $\lambda$)
%and \reftab{cubpoly} ($\lambda = \pm1$);
%for larger $n$ up to 8,
%we have saved the data
  are listed on the website in \refsec{end}.
%
The representative $r$ values are listed in \reftab{crval}.
%
%For complex $r$ and $\vx$,
%we have plotted \reffig{cubbulb} for regions of stability.









\begin{table}[h]\footnotesize
%\caption{
\tbl{
  Smallest positive $r$ at the onset and bifurcation points
  of the $n$-cycles of the cubic map.
  }
%\begin{center}
{
\begin{tabular}{l l l l}
%\begin{tabularx}{\textwidth}{
%  >{\hsize=0.5\hsize\centering\arraybackslash}X
%  >{\hsize=1.6\hsize}X
%  >{\hsize=1.6\hsize}X
%  >{\hsize=0.3\hsize\raggedright\arraybackslash}X
%}
%\begin{tabular*}{\textwidth}{l l l l c | c l l l l}
\hline
  $n^\dagger$
& Onset$^\ddagger$
& Bifurcation$^\ddagger$
& \#$^*$ \\
\hline
$1$     & $1.0000000000_1$      &  $2.0000000000_1$       & 2   \\
$2'$    & $2.0000000000_1$      &  $2.2360679775_2$       & 2   \\
$3$     & $2.4504409645_{8}$    &  $2.4608286739_{12}$    & 4   \\
$4$     & $2.5478350393_{22}$   &  $2.5488312193_{32}$    & 8   \\
$4''$   & $2.2360679775_{2}$    &  $2.2880317545_{32}$    & 2   \\
$5$     & $2.3939250274_{112}$  &  $2.3957922744_{120}$   & 24  \\
$6$     & $2.3334877526_{304}$  &  $2.3355337580_{336}$   & 56  \\
$6'$    & $2.4608286739_{12}$   &  $2.4657090579_{336}$   & 4   \\
$7$     & $2.3729872678_{1080}$ &  $2.3732727868_{1092}$  & 156 \\
$8$     & $2.3525990555_{3108}$ &  $2.3527637793_{3200}$  & 400 \\
$8'$    & $2.5488312193_{32}$   &  $2.5493247379_{3200}$  & 8   \\
$8'''$  & $2.2880317545_{32}$   &  $2.2992279397_{3200}$  & 2   \\
\hline
\multicolumn{4}{p{\linewidth}}{
$^\dagger$
  $\,'$, $\,''$, or $\,'''$ means that the cycle is undergoing
    the first, second, or third successive period-doubling, respectively.
} \\
\multicolumn{4}{p{\linewidth}}{
$^\ddagger$
  The subscripts are the degrees of the corresponding minimal polynomial.
} \\
\multicolumn{4}{p{\linewidth}}{
$^*$
  The number of similar cycles
    (for $n > 1$, only half of them have positive $r$).
} \\
\hline
\end{tabular}
%\end{tabularx}
%\end{tabular*}
%\end{center}
\label{tab:crval}
}
\end{table}


Similar to the logistic-map case,
the number of the even cube-free cyclic polynomials
is equal to the number of even \emph{ternary} necklaces,
in which each bead can assume
0, 1, or 2, with the total being even.
%
The number of ternary necklaces is
$N_3(n) = \frac{1}{n} \sum_{d|n} \varphi(n/d) 3^d$,
%\begin{theorem}
while the even ones sum to
%or the cube-free cyclic polynomials for the cubic map of even degrees in $\vx$
%is given by
\begin{equation}
  N_e(n) = \frac{1}{n} \sum_{c d = n} \varphi(c)
    \left[
      3^d - \odd(c) \frac{3^d-1}{2}
    \right],
\label{eq:cubnecklace}
\end{equation}
where
$\odd(c)$ %$\equiv \big[1 - (-1)^c\big]/2$
is 1 if $c$ is odd or 0 if even.
%\label{thm:cubnecklace}
%\end{theorem}
%
Since $N_e \approx N_3/2$, using the parity saves about half of the equations
  in computation.
%
$N_e(n) = $ 2, 4, 6, 14, 26, 68, 158, 424, \dots, from $n = 1$.
%



The characteristic polynomial $A_n(r, \lambda)$
  from the determinant equation
  has a degree $N_e(n)$ in $\lambda$.
Again, it encompasses the factors for the $n$-cycles
  and the shorter $d$-cycles, as long as $d|n$.
The minimal polynomial for the $n$-cycles can be obtained
  by \refthm{primfac} with proper substitutions
  $R\rightarrow r$,
    $A_n(R, \lambda) \rightarrow A_n(r, \lambda)$,
    etc.
%The degree of the polynomial,
The degree in $\lambda$ of
the minimal polynomial $P_n(r, \lambda)$ of the $n$-cycles,
  after some algebra,
  is given by
%
%
%\begin{theorem}
%The degree in $\lambda$ of
%the minimal polynomial $P_n(r, \lambda)$ of the $n$-cycles is
\begin{equation}
  L_e(n) = \frac{1}{n} \sum_{cd = n} \mu(c)
    \left[
      1 + \odd(c)\frac{3^d-1}{2}
    \right],
\label{eq:cublyndon}
\end{equation}
%where $\odd(n)$ %$= [1 - (-1)^n]/2$
%is $1$ for an odd $n$,
%but $0$ for an even $n$.
%\label{thm:cublyndon}
%\end{theorem}
%
%
For the first few values,
  $L_e(n)$ = 2, 2, 4, 10, 24, 60, 156, 410, \ldots, starting from $n = 1$.
%
This is also the number of the $n$-cycles \cite{hao, hao2}.
Note that, for $n>1$, $x_k$ in half of the cycles
  with negative $r$ are imaginary.
However, in a closely-related map
$z_{k+1} = r \, z_k \, (1 - {z_k}^2)$,
which differs from \refeq{cubic} by $x_k = \sqrt{r} z_k$,
$z_k$ in the negative-$r$ cycles are real.
%
%
Following similar steps in \refsec{degR},
we find the corresponding degrees in $r$
of the characteristic polynomial $A_n(r, \lambda)$
  and minimal polynomial $P_n(r, \lambda)$ of the $n$-cycles
are $n N_e(n)$ and $n L_e(n)$, respectively.



%\subsection{\label{sec:oddcycle}Odd-cycles}

%
%
\begin{figure}[h]
  \begin{center}
  \begin{minipage}{\linewidth}
        \includegraphics[angle=-90, width=\linewidth]{oddcycle.pdf}
  \end{minipage}%
  \end{center}
  \caption{\label{fig:oddcycle}
  Odd-cycles of the cubic map.}
\end{figure}
%
%

The symmetry $f(-x) = -f(x)$ permits
  a particular \emph{odd-cycle},
  see \reffig{oddcycle} for examples.
If $x_{n+1} = -x_1$ has a solution,
  then $x_1, \ldots, x_{n}, -x_1, \ldots, -x_{n}$
  is a $2n$-cycle, for $x_{2n+1} = - x_{n+1} = x_1$.
%
The existence of odd-cycles is why we used even, but not odd, cyclic polynomials
  in the solution.
An odd $2n$-cyclic polynomials is also always zero
  in an $n$-odd-cycle,
e.g., $C_1 = x_1 + x_2 + \dots + x_n - x_1 - x_2 - \dots - x_n = 0$;
and the zero determinant condition \refeq{xr}
  would omit the $n$-odd-cycles from the solution.


Since the $n$-odd-cycles satisfy a polynomial of lower degree,
  the minimal polynomial $P_{2n}(r, \lambda)$
  can always be factorized.
%
Suppose $\Lambda^\odd(\vx) \equiv \prod_{k = 1}^{n} f'(x_k)$ in the odd-cycle
 satisfies $P_{n}^\odd(r, \lambda^\odd) = 0$
\big[where $\lambda^\odd$ is the value of $\Lambda^\odd(\vx)$,
and $\lambda^\odd = \pm\sqrt \lambda$\,\big],
then $P^\odd_{n}(r, \sqrt \lambda) P^\odd_{n/2}(r, -\sqrt \lambda)$
is a factor of $P_{n}(r, \lambda)$.
%
For example, by solving the $n = 2$ odd-cycle,
  we have $- x_1 = r x_1 - {x_1}^3$, or ${x_1}^2 = r + 1$.
Since $\lambda^\odd = r - 3 {x_1}^2$,
$P^\odd_1(r, \lambda^\odd) = \lambda^\odd + 2 r + 3$.
%
The minimal polynomial of 2-cycles
  $P_2(r,\lambda) = -\big[(2r+3)^2 - \lambda\big]\,(\lambda + 2r^2-9)$,
  indeed contains
  $P^\odd_1(r, \sqrt \lambda) \, P^\odd_1(r, -\sqrt \lambda)
   = (\sqrt \lambda + 2 r + 3)(-\sqrt \lambda + 2 r + 3)$
  as a factor.




\section{\label{sec:end}Summary and discussions}


We now summarize the algorithm for a one-dimensional polynomial map.
%
First, we list \refeqs{xcp}  with $\Lambda(\vx) = \prod_{k=1}^n f'(x_k)$.
This step populates elements of the matrix $\vct T(r)$,
where $r$ is the parameter of the map.
%
The determinant $A_n(r, \lambda) = \big|\lambda \, \vct I - \vct T(r)\big|$,
with $\lambda$ being $+1$ and $-1$, then gives the characteristic
polynomial at onset and bifurcation points, respectively.
%
To filter out factors for the shorter $d$-cycles with $d|n$,
  we repeat the process for other divisors $d$ of $n$
  and then apply \req{primfac}.
%


When implemented on a computer,
it is often helpful to evaluate $A_n(r,\lambda)$ by Lagrange interpolation,
that is, we evaluate $A_n(r, \lambda)$ at a few different $r$,
e.g., $r = 0, \pm1, \pm2,\ldots$, then piece them together
to a polynomial.
The strategy also allows a trivial parallelization.



The algorithm (implemented as a Mathematica program)
was quite efficient.
%
For the logistic map, the bifurcation point for $n = 8$
took three seconds to compute on a desktop computer
(single core, Intel\textsuperscript{\textregistered} Dual-Core CPU 2.50GHz).
%
In comparison,
  the same problem took roughly 5.5 hours \cite{kk1}
  using Gr\"obner basis
  and 44 minutes in a later study \cite{lewis}.
To be fair, using the latest Magma, computing the Gr\"obner basis
  took 81 and 14 minutes, on the same machine
  for \refeq{logmap} and \refeq{logmaps}, respectively;
even so, our approach still had a 200-fold speed-up.




%
The exact polynomials of these maps are generally too large to print on paper,
  e.g., the polynomial for the logistic map with $n = 13$
    takes roughly seven megabytes to write down.
%
We therefore save the polynomials and programs of the three maps on the web site
http://logperiod.appspot.com.


%Compared with other approaches \cite{hitzl, gordon, bailey, kk1, hao, hao2}
%it appeared to be simpler and more transparent,
%as it only requires the computation of a determinant.



%\section*{Acknowledgements}
\nonumsection{Acknowledgments} \noindent
I thank Drs. T. Gilbert and Y. Mei for helpful communications.
Computing time on the Shared University Grid at Rice,
funded by NSF under Grant EIA-0216467, is gratefully acknowledged.

%\appendix{}

% ws-ijbc dislikes the following
%\appendix


%\appendixpage



%\appendix{\label{apd:per4}Derivation of the 4-cycles}
%
%
%
%The polynomials for the 4-cycles has a compact derivation.
%We first list the explicit equations:
%%
%%
%%
%\begin{subequations}
%\label{eq:x4}
%\begin{align}
%  x_2 &= R - {x_1}^2, \\
%  x_3 &= R - {x_2}^2, \\
%  x_4 &= R - {x_3}^2, \\
%  x_1 &= R - {x_4}^2,
%\end{align}
%\end{subequations}
%%
%$\big[\mathrm{\refeqsub{x4}{a}} - \mathrm{\refeqsub{x4}{c}}\big]
%\times
%\big[\mathrm{\refeqsub{x4}{b}} - \mathrm{\refeqsub{x4}{d}}\big]$
%yields $1 + (x_1 + x_3) (x_2 + x_4) = 0$,
%since $x_1 \ne x_3, x_2 \ne x_4$.
%%
%Hence, with
%$y_1 \equiv x_1 + x_3$, $y_2 \equiv x_2 + x_4$,
%$z \equiv y_1 + y_2$,
%we have
%%
%%
%%
%\begin{subequations}
%\begin{align}
%y_1 y_2           &= -1, \\
%{y_1}^2 + {y_2}^2 &= z^2 - 2 \, y_1 y_2 = z^2 + 2, \\
%{y_1}^3 + {y_2}^3 &= z^3 - 3 \, y_1 y_2 \, z = z^3 + 3 z.
%\end{align}
%\label{eq:ypow4}
%\end{subequations}
%%
%%
%Multiplying \refeqsub{x4}{a} by $x_1$ or $x_3$,
%then summing over cyclic versions yields
%\begin{subequations}
%\begin{align}
%y_1 y_2 &= R z - \big[({x_1}^3 + {x_3}^3) + ({x_2}^3 + {x_4}^3)\big],\\
%y_1 y_2 &= R z - \big[x_1 x_3 \, y_1 + x_2 x_4 \, y_2\big].
%\end{align}
%\label{eq:p4q}
%\end{subequations}
%%
%From
%$\mathrm{\refeqsub{p4q}{a}} + 3 \times \mathrm{\refeqsub{p4q}{b}}$,
%we have
%$4 \, y_1 y_2 = 4 R z - ({y_1}^3 + {y_2}^3)$,
%%
%and by \refeqs{ypow4},
%\begin{equation}
%  z^3 - (4 R - 3) z - 4 = 0.
%  \label{eq:xr4s}
%\end{equation}
%%
%%
%Since
%$2 x_1 x_3 = {y_1}^2 - ({x_1}^2 + {x_3}^2)
%= {y_1}^2 - 2 R + y_2$,
%and $2 x_2 x_4 = {y_2}^2 - 2 R + y_1$,
%%
%%Using \refeqs{ypow4} to simplify the right hand side, we find
%\begin{equation}
%  X \equiv x_1 x_2 x_3 x_4 = \tfrac{1}{2} R z(1 - z) + (R^2 - R + 1),
%\label{eq:der4}
%\end{equation}
%where we have used \refeqs{ypow4} and \req{xr4s} to simplify the result.
%%$X=1/16$ and $-1/16$ at the onset and bifurcation points, respectively.
%%
%Dividing the polynomial in \refeq{xr4s} by that in \refeq{der4}
%yields $z = (R^2-3R-X+1)/(R^2-R+X-1)$,
%and plugging it back to \refeq{der4} gives
%   $R^6 -3 R^5
%  + (3 + X) (R^4 - R^3)
%  + (1 - X) (2 + X) R^2
%  + (1 - X)^3 = 0$,
%which is the same as the sextic factor of \refeq{xr4}
%  with $X = \lambda/16$.
%
%








%\section{\label{apd:cublyndon}Proof of \refthm{cublyndon}}












\begin{thebibliography}{50}


\bibitem[May(1976)]{may}
  May, R. H. [1976]
  ``Simple mathematical models with very complicated dynamics,''
  {\it Nature}
  \textbf{261},
  459--467.

\bibitem[Strogatz(1994)]{strogatz}
  Strogatz, S. H. [1994]
  {\it Nonlinear Dynamics and Chaos}
  (Addison-Wesley. Reading, MA).

\bibitem[Mandelbrot(1980)]{mandelbrot}
  Mandelbrot, B. B. [1980]
  ``Fractal aspects of the iteration of $z \rightarrow \lambda z(1-z)$
      for complex $\lambda$ and $z$,''
  {\it Ann. NY Acad. Sci.}
  \textbf{357},
  249--259.

\bibitem[Brown(1981)]{brown1}
  Brown, A. [1981]
  ``Equations for periodic solutions of a logistic difference equation,''
  {\it J. Austral. Math. Soc. Ser. B}
  \textbf{23},
  78--94
  (1981).

\bibitem[Brown(1982)]{brown2}
  Brown, A. [1982]
  ``Solutions of period seven for a logistic difference equation,''
  {\it Bull. Austral. Math. Soc.}
  \textbf{26},
  263--284.

\bibitem[Saha \& Strogatz(1995)]{saha}
  Saha, P. \& Strogatz, S. H. [1995]
  ``The birth of period three,''
  {\it Mathematics Magazine}
  \textbf{68},
  42--47.

\bibitem[Kotsireas \& Karamanos(2004)]{kk1}
  Kotsireas, I. S. \& Karamanos, K. [2004]
  ``Exact computation of the bifurcation point $B_4$ of the logistic map
  and  the Bailey-Broadhurst conjectures,''
  {\it International Journal of Bifurcation and Chaos}
  \textbf{14},
  2417--2423.

\bibitem[Burm \& Fishback(2001)]{burm}
  Burm, J. \& Fishback, P. [2001]
  ``Period-3 orbits via Sylvester's theorem and resultants,''
  {\it Mathematics Magazine}
  \textbf{74},
  47--51.

\bibitem[Stephenson(1991)]{stephenson1}
  Stephenson, J. [1991]
  ``Formulae for cycles in the Mandelbrot set,''
  {\it Physica A}
  \textbf{177},
  416--420.

\bibitem[Stephenson(1992)]{stephenson2}
  Stephenson, J. [1992]
  ``Formulae for cycles in the Mandelbrot set II,''
  {\it Physica A}
  \textbf{190},
  104--116.

\bibitem[Stephenson(1992)]{stephenson3}
  Stephenson, J. [1992]
  ``Formulae for cycles in the Mandelbrot set III,''
  {\it Physica A}
  \textbf{190},
  117--129.

\bibitem[Bechhoefer(1996)]{bechhoefer}
  Bechhoefer, J. [1996]
  ``The birth of period 3, revisited,''
  {\it Mathematics Magazine}
  \textbf{69},
  115--118.

\bibitem[Gordon(1996)]{gordon}
  Gordon, W. B. [1996]
  ``Period three trajectories of the logistic map,''
  {\it Mathematics Magazine}
  \textbf{69},
  118--120.

\bibitem[Zhang(2010)]{zhang}
  Zhang, C. [2010]
  ``Period three begins,''
  {\it Mathematics Magazine}
  \textbf{83},
  295--297.

%\bibitem[Bailey \& Broadhurst(2000)]{bailey1}
%  Bailey, D. H. \& Broadhurst, D. J. [2000]
%  ``Parallel integer relation detection: techniques and applications,''
%  {\it Mathematics of Computation}
%  \textbf{70},
%  1719--1736.

\bibitem[Bailey {\it et al.}(2006)]{bailey}
  Bailey, D. H., Borwein, J. M., Kapour, V. \& Weisstein, E. W. [2006]
  ``Ten problems in experimental mathematics,''
  {\it The American Mathematical Monthly}
  \textbf{113},
  481--509.

\bibitem[Lewis(2008)]{lewis}
  Lewis, R. H. [2008]
  ``Heuristics to accelerate the Dixon resultant,''
  {\it Mathematics and Computers in Simulation}
  \textbf{77},
  400--407.

\bibitem[Hardy \& Wright(2008)]{hardy}
  Hardy, G. H. \& Wright, E. M. [2008]
  {\it An Introduction to the Theory of Numbers}
  (Oxford University Press, USA).

\bibitem[Hao(1989)]{hao}
  Hao, B.-L. [1989]
  {\it Elementary Symbolic Dynamics and Chaos in  Dissipative Systems}
  (World Scientific, Singapore).

\bibitem[Hao(2000)]{hao2}
  Hao, B.-L. [2000]
  ``Number of periodic orbits in continuous maps of
  the interval complete solution of the counting problem,''
  {\it Annals of Combinatorics}
  \textbf{4},
  339--346.

\bibitem[Lutzky(1988)]{lutzky}
  Lutzky, M. [1988]
  ``Counting stable cycles in unimodal iterations,''
  {\it Physics Letters A}
  \textbf{131},
  248--250.

\bibitem[Mira(1987)]{mira}
  Mira, C. [1987]
  {\it Chaotic Dynamics}
  (World Scientific, Singapore).

\bibitem[Blackhurst(2011)]{blackhurst}
  Blackhurst, J. [2011]
  ``Polynomials of the bifurcation points of the logistic map,''
  {\it International Journal of Bifurcation and Chaos}
  \textbf{21},
  1869--1877.

\bibitem[H\'enon(1976)]{henon}
  H\'enon, M. [1976]
  ``A two-dimensional mapping with a strange attractor,''
  {\it Communications in Mathematical Physics}
  \textbf{50},
  69--77.

\bibitem[Hitzl(1985)]{hitzl}
  Hitzl, D. L. \& Zele, F. [1985]
  ``An exploration of the H\'enon quadratic map,''
  {\it Physica D}
  \textbf{14},
  305--326.

\bibitem[Brown(1984a)]{brown3}
  Brown, A. [1984a]
  ``Solutions of period three for a non-linear difference equation,''
  {\it J. Austral. Math. Soc. Ser. B}
  \textbf{25},
  451--462.

\bibitem[Brown(1984b)]{brown4}
  Brown, A. [1984b]
  ``Solutions of period four for a non-linear difference equation,''
  {\it J. Austral. Math. Soc. Ser. B}
  \textbf{26},
  146--164.


\end{thebibliography}

\end{document}


