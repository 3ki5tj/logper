\documentclass{ws-ijbc}
\usepackage{multirow}
\usepackage{centernot}
%\usepackage{hyperref}
\usepackage{graphicx}
\usepackage{epstopdf}
\usepackage{calc}
\usepackage{cancel} % strike out (cancel) in math mode
\usepackage{ws-rotating}
%\usepackage{multicol}
\begin{document}



\newcommand{\odd}{\mathrm{odd}}
\newcommand{\even}{\mathrm{even}}

% bold font for vectors
\newcommand{\vct}[1]{\mathbf{#1}}
% the cycle and sum operator
%\newcommand{\C}{\mathcal{C}}


% center column in tabularx
\newcolumntype{C}{>{\centering\arraybackslash}X}
\newcolumntype{L}{>{\raggedright\arraybackslash}X}
\newcolumntype{R}{>{\raggedleft\arraybackslash}X}


\newcommand{\vx}[1]{\vct x_{#1}}
\newcommand{\vy}[1]{\vct y_{#1}}
\newcommand{\vz}[1]{\vct z_{#1}}
\newcommand{\Tr}{\mathrm{Tr}}
% basis set of cyclic polynomials
\newcommand{\Pset}{\mathcal P}
\newcommand{\NB}{N_\Pset}
\newcommand{\LB}{L_\Pset}
\newcommand{\lam}{(\lambda)}
\newcommand{\Rlam}{(R, \lambda)}
\newcommand{\rlam}{(r, \lambda)}

%\newcommand{\eq}{}
%\newcommand{\eqs}{}
\newcommand{\eq}{Eq.}
\newcommand{\eqs}{Eqs.}
\newcommand{\req}[1]{(\ref{eq:#1})}
\newcommand{\refeq}[1]{\eq\,\req{#1}}
\newcommand{\refeqs}[1]{\eqs\,\req{#1}}
\newcommand{\reqsub}[2]{(\ref{eq:#1}#2)}
\newcommand{\refeqsub}[2]{\eq\,\reqsub{#1}{#2}}
\newcommand{\refeqssub}[2]{\eqs\,\reqsub{#1}{#2}}

\newcommand{\refthm}[1]{Theorem \ref{thm:#1}}
\newcommand{\refthms}[1]{Theorems \ref{thm:#1}}
\newcommand{\refsec}[1]{Section \ref{sec:#1}}
\newcommand{\refsecs}[1]{Sections \ref{sec:#1}}
\newcommand{\refapd}[1]{Appendix \ref{apd:#1}}
\newcommand{\reftab}[1]{Table \ref{tab:#1}}
\newcommand{\reftabs}[1]{Tables \ref{tab:#1}}
\newcommand{\reffig}[1]{Fig. \ref{fig:#1}}
\newcommand{\reffigs}[1]{Figs. \ref{fig:#1}}

% ws-ijbc doesn't like the following lines
%\newtheorem{theorem}{Theorem}
%\newenvironment{remark}[1][1]%
%{\par\noindent\textbf{Remark #1.} }{\medskip}


\catchline{}{}{}{}{}




\title{Cycles of the logistic map}
\author{Cheng Zhang}
\address{Applied Physics Program \& Department of Bioengineering,
Rice University, Houston, TX 77005, USA}

\maketitle


\begin{history}
\received{}
\end{history}


\begin{abstract}
The onset and bifurcation points of the $n$-cycles of
  several polynomial maps are located
  through a characteristic equation
  connecting cyclic polynomials of the cycle points.
The polynomials satisfied by the parameters
  of the logistic, H\'enon, and cubic maps
  at the onset and bifurcation points are obtained
  for $n$ up to 14, 9, and 9,
  respectively.
\end{abstract}

\keywords{Logistic map; H\'enon map; cubic map; cycles; exact polynomials.}


%\maketitle

% switch text to two-column
%\begin{multicols}{2}
\twocolumn

%
%
\section{Introduction}
%
%

Consider the logistic map \cite{may, strogatz}:
%
\begin{equation}
  x_{k+1} = f(x_k) \equiv r \, x_k \, ( 1 - x_k ).
\label{eq:logmap}
\end{equation}
%
%The iterated sequence $x_1$,
%  $x_2 = f(x_1)$,
%  $x_3 = f(x_2)$,
%  $\ldots$,
%can be conveniently visualized on the cobweb plot,
%  as shown in \reffig{cobweb}.
%%
%Starting from $(x_1, x_1)$ on the diagonal,
%  each vertical arrow takes $(x_k, x_k)$ to $(x_k, y)$,
%  where $y = x_{k+1} = f(x_k)$;
%by the following horizontal arrow,
%  it returns to the diagonal at
%  $(y, y) = (x_{k+1}, x_{k+1})$,
%  which starts the next iteration.
%
%
%\begin{figure}[h]
%  \begin{center}
%  \begin{minipage}{\linewidth}
%        \includegraphics[angle=-90, width=\linewidth]{cobweb.pdf}
%  \end{minipage}%
%  \end{center}
%  \caption{\label{fig:cobweb}
%  Cobweb plot of the logistic map.}
%\end{figure}
%
%
After many iterations, there can be three ultimate patterns:
  (i) a fixed point (possibly at the infinity),
      %e.g., \reffig{cobweb}(a);
  (ii) a periodic cycle,
      %e.g., \reffigs{cobweb}(b)-(e);
or
  (iii) a chaotic or aperiodic trajectory.
      %e.g., \reffig{cobweb}(f).
We only discuss the first two cases here.


A fixed point is a solution of $x^* = f(x^*)$.
%
A stable fixed point requires that a small deviation from $x^*$
  gets reduced in magnitude after an iteration of \refeq{logmap}.
%
Since $f$ is differentiable,
  the condition is equivalent to $|f'(x^*)| \le 1$
  \cite{strogatz, hao}.
%
An $n$-cycle is a sequence $x_1, \dots, x_n$,
  such that $n$ is the smallest positive integer
  that satisfies $x_1 = x_{n+1} = f^n(x_1)$,
  where $f^n$ is the $n$th iterate of $f$,
  e.g., $f^3(x) = f(f(f(x)))$.
By definition, any $x_k$ in an $n$-cycle of $f$ is a fixed point of $f^n$
(but the latter also permits an $x_k$ in a $d$-cycle with $d|n$).
%
%\big[but the converse is untrue, for a fixed point of $f^n$
%  can also be a fixed point of $f^d$ for a divisor $d$ of $n$:
%  if $f^d(x) = x$, then $f^n(x) = f^d(\cdots f^d(x)\cdots) = x$\big].
%
%
Thus, the stability of a cycle can be defined
  according to that of the corresponding fixed point of $f^n$:
  a stable cycle requires
  $\big| \frac {d} {dx_1} f^n(x_1) \big| \le 1$,
  or by the chain rule,
%
%
%
\begin{equation}
  \Big| f'(x_n) \cdots f'(x_1) \Big| \le 1,
\label{eq:der}
\end{equation}
%
%
The onset and bifurcation points
  are defined at the loci
  where $\frac {d} {dx_1} f^n(x_1)$ reaches $+1$ and $-1$,
  respectively \cite{strogatz, hao}.
%
%
%
The cycle and its stability depend on the parameter $r$,
%
and a stable cycle corresponds to a window
  between the onset point $r_a$ and bifurcation point $r_b$.
%
There are generally multiple such windows for a single $n$,
  but all $r_a$ are zeros of a polynomial,
  and all $r_b$ another.
%
Our aim is thus to find the two polynomials for a given $n$.
%
The problem has been tackled either by hand for small $n$
\cite{brown1, brown2, stephenson1, stephenson2, stephenson3,
      saha, bechhoefer, gordon, burm, zhang}
or by general computer algebra methods \cite{bailey1, bailey2, kk1, lewis}.
%
Below we present a specialized algorithm to do so.
%
%
%
%The window boundaries can be classified
%  to onset and bifurcation points according to \refeq{der};
%Here, if $r > 0$, then
%  $r_a$ ($r_b$) are the onset (bifurcation) points.
%







\section{\label{sec:logmap}Logistic map}



To simplify the calculation, we first change variables \cite{mandelbrot, brown1} as
%
%\begin{equation}
\[
    x^{\mathrm{new}} \equiv r(x^{\mathrm{old}} - 1/2),
    \quad R \equiv r(r-2)/4,
\]
%\label{eq:chvars}
%\end{equation}
%
and rewrite the map, in terms of $x^{\mathrm{new}}$, as
%
\begin{equation}
  x_{k+1} = f(x_k) \equiv R - {x_k}^2.
  %\tag{$\ref{eq:logmap}'$}
\label{eq:logmaps}
\end{equation}
%
%\refeq{logmaps} simplifies \refeq{logmap} as $x_k$ now only occurs once.
%
We will solve the cycle boundaries as zeros of some polynomials of $R$,
and the corresponding polynomials for $r$ can be obtained by $R \rightarrow r(r-2)/4$.
%
%the numerical $r$ values can always be obtained from $r = 1 + \sqrt{1 + 4R}$.
%






\subsection{Plan}

Since the $n$-cycles points constitute
  a subset of the fixed points of $f^n$,
we will first find
  the polynomials of marginal stability
  for the fixed points of $f^n$
  (\refsecs{cyclic}-\ref{sec:examples}).
The polynomials contain
  extraneous factors contributed by short $d$-cycles with $d|n$,
  which are also fixed points of $f^n$.
The factors are removed in \refsec{primfac}.


At first glance, the polynomials of the fixed points of $f^n$
  can be obtained by brute force.
We can solve
  \refeqs{logmaps} and express $x_1, \ldots, x_n$
  in terms of $R$,
  and then plug the solution into \refeq{der}
  to get a polynomial equation of $R$.
But since \refeqs{logmaps} are nonlinear,
  it is nontrivial to reduce the final equation of $R$
  into a polynomial one.
Nonetheless, on a computer, one can
  construct a polynomial Gr\"obner basis \cite{kk1},
  which gradually eliminates $x_k$ out of the $n+1$ equations.
If $x_2, \dots, x_n$ are eliminated first,
  there are only two polynomial equations left:
  $f^n(x_1) - x_1 = 0$
  and
  $\frac{d}{dx_1}f^n (x_1) \mp 1 = 0$
  [\refeq{der}];
  the final elimination of $x_1$ can then be accomplished
  by setting the resultant of the two polynomials
  to zero \cite{burm}.
  %leading to the desired polynomial equation of $R$.
%
These approaches, however, have not fully
  exploited the cyclic structure of \refeqs{logmaps},
  and can be improved by the following alternative.


Instead of solving \refeqs{logmaps} for $x_k$,
  we will derive a set of equations
  in terms of the cyclic polynomials $\{C_p\}$ of $x_k$.
  %(an example of a cyclic polynomial is
  %$x_1 x_2 + x_2 x_3 + \dots + x_n x_1$
  %with $n$ terms).
%
The advantage is that the new set of equations are
  \emph{homogeneous}, \emph{linear} and \emph{complete},
  i.e., there is an equation
  $\sum_q A_{pq}(R) C_q = 0$ for every cyclic polynomial $C_p$.
%
The matrix $\mathbf A(R)$ formed by the coefficients $A_{pq}(R)$
  must have a zero determinant,
  for the cyclic polynomials $\{C_p\}$ are not zeros altogether.
%
Thus, the zero-determinant condition $|\mathbf A(R)| = 0$ gives
  the required polynomial equation of $R$.
%whose roots contain all fixed points of $f^n$.
%




\subsection{\label{sec:cyclic}Cyclic polynomials}


A cyclic polynomial is an invariant
  under the cycling of variables
  $x_1 \rightarrow x_2, x_2 \rightarrow x_3,
  \ldots, x_n \rightarrow x_1$,
  e.g., $a = x_1 x_2 + x_2 x_3 + x_3 x_4 + x_4 x_1$
  and $b = x_1 x_3 + x_2 x_4$ for $n = 4$.
%
A cyclic polynomial is not necessarily symmetric,
  that is, invariant under the exchange of any two $x_i$ and $x_{j}$,
  e.g., the above $a$ and $b$ are not symmetric, but $a + b$ is.


We call a cyclic polynomial simple
  if it is equal to the sum of
  all distinct cyclic versions of
  a unit-coefficient monomial of $x_k$.
%
The monomial is called a generator of the cyclic polynomial.
%
For example, the above $a$ and $b$ are simple,
  and their generators are $x_1 x_2$ and $x_1 x_3$, respectively.
%
A simple cyclic polynomial
  $C_p(\vx{n})$, with $\vx{n} \equiv \{x_1, \ldots, x_n\}$,
can be labeled as follows.
%
If the monomial generator is
  ${x_1}^{e_1} {x_2}^{e_2} \dots {x_n}^{e_n}$,
  we define the label $p$ as the sequence of indices with
  $e_1$ 1's, $e_2$ 2's, \ldots, $e_n$ $n$'s.
%
For example,
$C_{12}(\vx{n})  = x_1 x_2 + x_2 x_3 + \dots + x_n x_1$
and
$C_{112}(\vx{n}) = {x_1}^2 x_2 + {x_2}^2 x_3 + \dots + {x_n}^2 x_1$.
%
Below, we may drop %the subscript ``$_n$'' or even the argument
  ``$(\vx{n})$'' when it is clear.
In case the cyclic polynomial has several generators,
  we pick the one with the smallest label $p$
  in the sense of lexicographic order,
  e.g., we choose $x_1 x_2$ over $x_2 x_3$ as the generator of $C_{12}(\vx{n})$,
  because ``$12$'' $<$ ``$23$.''
%
Finally, we add $C_0(\vx{n}) \equiv 1$.
%
%
%




\subsection{Square-free reduction}




A cyclic polynomial is square-free
  if it has no square or higher powers of $x_k$,
e.g., $C_{12} = x_1 x_2 + \cdots$ is square-free,
  $C_{112} = {x_1}^2 x_2 + \cdots$ is not.
%see \reftab{sqrfreepoly} for more examples.
The label $p$ of a square-free polynomial
  has no repeated index.
We denote the set of all square-free $p$ by
$\Pset = \{0, 1, 12, 13, \ldots, 123, 124, \ldots, 12\cdots n\}$,
and its size by $\NB$ [cf. \refeq{necklace} below for the formula].
% is the number of square-free cyclic polynomials.
%
%
The set of the simple square-free cyclic polynomials
  furnishes a basis for expanding cyclic polynomials:



\begin{theorem}
  For the logistic map \refeq{logmaps},
  any cyclic polynomial $K(\vx{n})$
  formed by the $n$-cycle points
  $\vx{n} = \{x_1, \ldots, x_n\}$
  is a linear combination of
  the simple square-free cyclic polynomials $C_p(\vx{n})$:
\[
  K(\vx{n}) = \sum_{p \in \Pset} f_p(R) \, C_p(\vx{n}),
\]
  where $\Pset$ % = \{0, 1, 12, 13, \ldots, 12\dots n\}$ is
  is the set of indices of all square-free cyclic polynomials,
  and $f_p(R)$ are polynomials of $R$.
  \label{thm:sqrfree}
\end{theorem}
%
%
\begin{proof}
Consider the following square-free reduction.
%
Given a cyclic polynomial $K(\vx{n})$,
  we recursively apply \refeq{logmaps} as
  $x_k^2 \rightarrow R - x_{k+1}$,
  until all squares or higher powers of $x_k$ are eliminated.
The process will not last indefinitely
  for each substitution reduces the degree in $x_k$ %(no matter which $k$)
  by one.
Since both the replacement rules and the original polynomial are cyclic,
  so is the final square-free polynomial.
Terms involving no $x_k$
  are collected to serve as the coefficient before $C_0(\vx{n}) \equiv 1$.
The coefficients $f_p$ are polynomials of $R$,
  for $R$ is the only new variable introduced by the substitutions.
\end{proof}
%
%
For example, the cyclic polynomial
  $K(\vx{2}) = {x_1}^2 \, x_2 + {x_2}^2 \, x_1$
can be written as $K = (R + 1) \, C_{1} - 2 R \, C_0$
for
${x_1}^2 \, x_2 = R \, x_2 - {x_2}^2 = R \, x_2 - R + x_1$ and
${x_2}^2 \, x_1 = R \, x_1 - R + x_2$.

\refthm{sqrfree}
  shows that any cyclic polynomial can be expanded
  as a combination of the square-free ones.
We will further show below that
  at the onset and bifurcation points,
  the $\NB$ square-free cyclic polynomials $C_p$ are themselves
  linearly connected by a characteristic equation
  furnished by \refeq{der},
  which is the key step of the solution.




\subsection{\label{sec:algo}Characteristic equation}


%The matrix of the connecting coefficients thus has a vanishing determinant.
%This condition furnishes the desired polynomial equation.


First, the derivative $\frac{d}{dx_1}f^n(x_1)$ is
  a cyclic polynomial of $\vx{n} = \{x_1, \ldots, x_n\}$:
%
%
\begin{equation}
  \Lambda(\vx{n})
   = f'(x_n) \dots f'(x_1)
   = (-2)^n x_1 \dots x_n.
\label{eq:logder}
\end{equation}
%
%
For any square-free cyclic polynomial $C_p(\vx{n})$,
the product $\Lambda(\vx{n}) \, C_p(\vx{n})$ is also cyclic,
  and can thus be expanded by \refthm{sqrfree} as
\begin{equation}
  \Lambda(\vx{n}) \, C_p(\vx{n}) = \sum_{q \in \Pset} T_{pq}(R) \, C_q(\vx{n}),
\label{eq:xcp0}
\end{equation}
where $T_{pq}(R)$ is a polynomial of $R$,
and $p, q \in \Pset$.



By \refeq{der},
  $\Lambda(\vx{n})$ equals $\lambda = +1$ ($-1$)
  at the onset (bifurcation) point,
  respectively;
  so \refeq{xcp0}
  becomes a homogeneous linear equation of
  $C_p(\vx{n})$:
\begin{equation}
  \lambda \, C_p(\vx{n}) = \sum_{q \in \Pset} T_{pq}(R) \, C_q(\vx{n}),
\label{eq:xcp}
\end{equation}
%
%
%where $\lambda$ is $+1$ and $-1$ at the onset and bifurcation points, respectively;
or in matrix form,
\begin{equation}
  \big[ \vct T(R) - \lambda \, \vct I \big] \, \vct C = 0,
\tag{\ref{eq:xcp}$'$}
\end{equation}
%
%
where
$\vct I$ is the $\NB \times \NB$ identity matrix,
$\vct T(R) = \{T_{pq}(R)\}$ is an $\NB \times \NB$ matrix,
and
$\vct C = \{C_p(\vx{n})\}$ is an $\NB$-dimensional column vector.




Since a set of homogeneous linear equations
  has a nonzero solution of $\vct C$
  only if the determinant of the coefficient matrix
  is zero, we have
\begin{equation}
  A_n\Rlam \equiv \big| \vct T(R) - \lambda \, \vct I \big| = 0.
\label{eq:xr}
\end{equation}
%
This determinant $A_n\Rlam$
  is the desired polynomial of $R$ and $\lambda$.
%
%The attached subscript $n$ allows us
%  to use the determinant later with
%  similar determinants of divisors $d$ of $n$.
%
To summarize, we have
\begin{theorem}
  At the onset and bifurcation points,
  the square-free cyclic polynomials $C_p(\vx{n})$
  of the cycle points $\vx{n} = \{x_k\}$
  are linearly related as
  $\lambda \, C_p(\vx{n}) = \sum_{q \in \Pset} T_{pq}(R) \, C_q(\vx{n})$,
  with $\lambda$ being $+1$ and $-1$, respectively,
  and $T_{pq}(R)$ being the coefficients
  from the square-free reduction of the cyclic polynomial
  $C_p(\vx{n}) \, \prod_k f'(x_k)$. %[\refeq{logder}].
  Thus, the fixed points of $f^n$
  are marginally stable when $R$ satisfies
  the characteristic equation
  $\big| \vct T(R) - \lambda \, \vct I \big| = 0$.
  \label{thm:main}
\end{theorem}

%\begin{remark}
%  With $\lambda = 0$,
%  the algorithm yields the polynomial of the superstable point,
%  at which the initial deviation from a cycle point vanishes
%  to the linear order after $n$ iterations of $f$.
%  But we have a better algorithm in this case:
%  since at least one of the $x_k$ is zero by \refeq{logder},
%  then $f^n(x_k = 0) = x_{n+k} = 0$, which provides
%  the desired equation of $R$ \cite{strogatz}.
%\end{remark}





\subsection{\label{sec:examples}Examples}




We illustrate the above algorithm on several small-$n$ cases.
%
%It is still helpful to have a mathematical software
%  at hand to verify certain steps
%  (e.g., in computing the determinants and their factorization).
%
%
For $n = 1$, % (the fixed point),
we have two simple square-free cyclic polynomials
$\{C_0, C_1\} = \{1, x_1\}$, with
$\Lambda = -2 \, x_1$. So
\[
  \Lambda
  \left( \begin{array}{c}
  C_0 \\
  C_1
  \end{array} \right)
  =
  \left( \begin{array}{cc}
  0     & -2 \\
  -2R   &  2
  \end{array}\right)
  \left( \begin{array}{c}
  C_0 \\
  C_1
  \end{array} \right).
\]
The characteristic equation (\ref{eq:xr}) reads
\begin{equation}
A_1\Rlam
  %=\left| \begin{array}{cc}
  %\lambda     & 2          \\
  %2 R         & \lambda - 2
  %\end{array} \right|
  \equiv -4R - 2\lambda + \lambda^2
  = 0.
  \tag{\ref{eq:xr}-1}
\label{eq:xr1}
\end{equation}
So $R = (\lambda^2 -2\lambda)/4$,
  so the fixed point becomes unstable at $R_b = 3/4$ where $\lambda = -1$.



For $n = 2$, we have %three square-free cyclic polynomials
$\{C_0, C_1, C_{12}\}
  = \{1,  x_1 + x_2, x_1 x_2 \}$
and
$\Lambda = 4 \, x_1 x_2$.
So
$\Lambda \, C_0 = 4 \, C_{12}$,
$\Lambda \, C_1 %= 4 \, {x_1}^2 \, x_2 + 4 \, x_1 \, {x_2}^2
  = -8 \, R \, C_0 + 4 \, (R + 1) \, C_1$
by the example after \refthm{sqrfree},
$\Lambda \, C_{12}
= 4 \, (R - x_2) (R - x_1) = 4 \, (R^2C_0-RC_1+C_{12})$,
and \refeq{xr} reads
%\[
%\Lambda
%  \left( \begin{array}{c}
%  C_0 \\
%  C_1 \\
%  C_{12}
%  \end{array} \right)
% =
%  \left( \begin{array}{ccc}
%  0           & 0         & 4 \\
%  -8R         & 4(R+1)    & 0 \\
%  4R^2        & -4R       & 4
%  \end{array} \right)
%  \left( \begin{array}{c}
%  C_0 \\
%  C_1 \\
%  C_{12}
%  \end{array} \right).
%\]
%
%
%
%The characteristic equation (\ref{eq:xr}) reads
%
%
%
{\small
\begin{align}
A_2\Rlam \equiv (4 - 4R - \lambda)
  \left[
    (4 R-\lambda)^2 - 4 \lambda
  \right] = 0.
  \tag{\ref{eq:xr}-2}
\label{eq:xr2}
\end{align}
}
%
%
%
Only the first factor $P_2\Rlam \equiv 4-4R -\lambda$
  corresponds to the 2-cycle  (cf. \refsec{primfac}).
%  while the second factor is from the fixed points
So $R = 1 - \lambda/4$,
and the onset ($\lambda = +1$) value is $R_a = 3/4$ %($r_a = 3$)
and the bifurcation ($\lambda = -1$) value is $R_b = 5/4$. %($r_b = 1+\sqrt 6$).
%
%Note that the onset point of the 2-cycle
%  is located at $R = 3/4$,
%  where the fixed point bifurcates \cite{strogatz}.
%  \big[compare \reffigs{cobweb}(a) and (b)\big].




%The $n = 3$ case is nontrivial and has sparked several different
%  elementary derivations
%  \cite{brown1, saha, bechhoefer, gordon, burm, zhang}.
%For the current algorithm,
%%the cyclic polynomials are
%%$C_0 = 1$,
%%$C_1 = x_1 + x_2 + x_3$,
%%$C_{12} = x_1 x_2 + x_2 x_3 + x_3 x_1$,
%%$C_{123} = x_1 x_2 x_3$,
%%and
%$\Lambda = -8 \, C_{123}$,
%%
%and the square-free reduction yields
%%
%\begin{small}
%\begin{align*}
%\Lambda
%\left( \begin{array}{c}
%  C_0 \\
%  C_1 \\
%  C_{12} \\
%  C_{123}
%\end{array} \right)
%=
%\left( \begin{array}{cccc}
% 0            & 0                 & 0               & -8 \\
% -24 R        & 8(R+1)            & -8R             & 0 \\
% 24 R^2       & -8R(R+2)          & 8(R+1)          & 0 \\
% -8R^3        & 8R^2              & -8R             & 8
%  \end{array} \right)
%\left( \begin{array}{c}
%  C_0 \\
%  C_1 \\
%  C_{12} \\
%  C_{123}
%\end{array} \right),
%\end{align*}
%\end{small}
%%
%%
%%
%and \refeq{xr} reads:
%\begin{align}
%  0
%&= -\big[
%  64R^3 - 128 R^2 - 8 (\lambda - 8) R - (\lambda -8)^2
%  \big] \notag \\
%&\qquad
%  \big(
%  \lambda^2 - 8\lambda - 24 R\lambda - 64 R^3
%  \big).
%  \tag{\ref{eq:xr}-3}
%\label{eq:xr3}
%\end{align}
%%
%Using the first factor, we find at the onset point
%  $\lambda = 1$,
%  $\left(
%  R-\frac74
%  \right)
%  \left(
%  R^2-\frac14 R + \frac{7}{16}
%  \right)=0$
%  and its only real solution is $R_a = 7/4$ ($r_a = 1+\sqrt 8$).
%%
%At the bifurcation point $\lambda = -1$,
%  the equation $R^3-2R^2 + \frac{9}{8} R -\frac{81}{64}=0$
%  yields
%  $R_b = \frac14
%      \left(
%      \frac{8}{3} +
%        \sqrt[3]{\frac{1915}{54} - \frac52\sqrt{201}}
%       +\sqrt[3]{\frac{1915}{54} + \frac52\sqrt{201}}
%      \right)$,
%whose corresponding $r = 1+\sqrt{1+4R}$
%is identical to that in \cite{gordon, burm}.
%%




For the $n = 4$ case,
%\cite{brown1, stephenson1, stephenson3, bailey1, bailey2, kk1, lewis},
%$C_0 = 1$,
%$C_1 = x_1 + \cdots + x_4$,
%$C_{12} = x_1 x_2 + \cdots + x_4 x_1$,
%$C_{13} = x_1 x_3 + x_2 x_4$,
%$C_{123} = x_1 x_2 x_3 + \cdots + x_4 x_1 x_2$,
%$C_{1234} = x_1 x_2 x_3 x_4$,
%and
%$\Lambda = 16 \, C_{1234}$.
%
we only list the final \refeq{xr}: %characteristic equation (\ref{eq:xr}):
%after the square-free reduction:
%(with six square-free cyclic polynomials):
%
\begin{small}
\begin{align}
A_4\Rlam
%& = \left|
%\begin{array}{cccccc}
% \lambda  & 0               & 0                 & 0     & 0     & -16 \\
% 64 R     & \lambda-16(R+1) & 16 R              & 0     & -16R  & 0 \\
% -64 R^2  & 16R(R+2)        & \lambda-16(R^2+1) & -32 R & 16R   & 0 \\
% -32 R^2  & 16R(R+1)        & -16R              & \lambda-16(R^2+1)  & 0 & 0 \\
% 64 R^3   & -16R^2(R+3)     & 16R(R+2)        & 32 R (R+1)       & \lambda-16(R+1) & 0 \\
% -16R^4   & 16R^3           & - 16R^2           & -16R^2            & 16R & \lambda - 16
%\end{array}
%\right | \nonumber  \\
& \equiv
    \Big[
    4096 R^6-12288 R^5 +
    256 (\lambda + 48) (R^4 - R^3) \notag \\
& -16 (\lambda + 32)(\lambda - 16) R^2
      -(\lambda - 16)^3 \Big] \notag \\
& \quad
    \big[(16 R^2 + \lambda)^2 - 16 (2R+1)^2 \lambda \big] \notag \\
& \quad
    \big[ 16 (R-1)^2 - \lambda \big]
 = 0.
\tag{\ref{eq:xr}-4}
\label{eq:xr4}
\end{align}
\end{small}
%
%
%
The first (sextic) factor, denoted as $P_4\Rlam$,
  is responsible for the 4-cycle.
%
At the onset point, % $\lambda = 1$.
we have %it can be factorized as
$ P_4(R, 1)
= (
    5 - 4R
  )
\, \big[
  (4R + 1)^2 + 4
  \big]
\, \big[
  108 - (4R - 3)^3
  \big]$,
%
%
%
with two real roots:
$R_a' = 5/4$
%\big($r_a' = 1+\sqrt{6} \approx 3.4495$\big)
for the period-doubled 2-cycle,
  %\big[compare \reffigs{cobweb}(b) and (d)\big],
and
$R_a = (3+\sqrt[3]{108})/4$
%\big($r_a = 1+\sqrt{4+\sqrt[3]{108}} \approx 3.9601$\big)
for an original 4-cycle. %\big[\reffig{cobweb}(e)\big].
%
At the bifurcation point, % ($\lambda = -1$),
we have $P_4(R, -1)
 = 4096 R^6 - 12288 R^5 + 12032 (R^4 - R^3)
  + 8432 R^2 + 4913$.
%which upon $R \rightarrow r(r-2)/4$ yields the same polynomial
%obtained previously
%\cite{stephenson1, stephenson3, bailey2, kk1, lewis}.
%
%The only two positive roots $r_b \approx 3.9608$
%and $r_b' \approx 3.5441$ of the sextic equation
%correspond to $r_a$ and $r_a'$ respectively.
%
%Since the 4-cycle problem were tackled mainly on computers
%  in the previous studies,
%  we also supply a compact elementary derivation
%  in the Appendix.
%
In the next section, we show how to systematically eliminate
the extraneous factors encountered in \refeqs{xr2} and \req{xr4}.





\subsection{\label{sec:primfac}Polynomials of $n$-cycles}


The characteristic polynomial $A_n\Rlam$ in \refeq{xr}
  is derived for the fixed points of $f^n$.
Thus, it encompasses not only the $n$-cycles,
  but also shorter $d$-cycles for divisors $d$ of $n$.
%
Fortunately, the contributions from cycles of different periods
  are separate factors of $A_n\Rlam$.
For a large $n$,
  the factor $P_n\Rlam$ of $n$-cycles only,
  or the \emph{$n$-cycle polynomial} below,
  can be recognized
  as the largest irreducible factor of $A_n\Rlam$.
%
More systematically, we have
%\begin{theorem}
%  The polynomial $P_n\Rlam$ of all $n$-cycles
%  is a factor of the characteristic polynomial $A_n\Rlam$,
%  and can be computed as
  \begin{equation}
    P_n\Rlam
    = \prod_{cd = n} B_d^c\Rlam^{\mu(c)},
  \label{eq:primfac}
  \end{equation}
where
  $B_d^c\Rlam \equiv \prod_{k=1}^c A_{d}(R, e^{2k\pi i/c} \lambda^{1/c})$
  with
  $\lambda^{1/c}$ being a complex $c$th root of $\lambda$.
Particularly, $B_n^1\rlam = A_n\rlam$.
The outer product runs over all divisors $d$ of $n$.
%\end{theorem}
%
%
%\begin{remark}
%Here, ``minimal'' means that the polynomial contains all zeros.
%It happens to be irreducible for a general $\lambda$.
%At $\lambda = 1$, the polynomial of $R$, however, can be factorized (see later).
%\end{remark}
%
%\begin{remark}
The M\"obius function $\mu(n)$ is $(-1)^k$
  if $n$ is the product of $k$ distinct primes,
  or 0 if $n$ is divisible by a square of a prime:
%Thus,
$\mu(n) = 1, -1, -1, 0, -1, 1, -1, \ldots,$
  from $n = 1$
%The $\mu(n)$ is useful for inversion:
%$g(n) = \sum_{d|n} \mu(n/d) \, h(d)$
%if and only if $h(n) = \sum_{d|n} g(d)$
 \cite{hardy}.
%\end{remark}
%



%\begin{remark}
%$B_{d,c}\lambda$ is a polynomial of $\lambda$.
Despite the radical argument $\lambda^{1/c}$,
  the product
  $B_d^c\Rlam = \prod_{k=1}^c A_d(R, e^{2k\pi i/c} \lambda^{1/c})$
  is a polynomial of $\lambda$,
  for it is invariant under
  $\lambda \rightarrow e^{2\pi i} \lambda$.
Also, $\deg_\lambda B_d^c\rlam = \deg_\lambda A_d\rlam$.
For convenience, we will drop the $R$ in ``$\Rlam$'' %or the arguments entirely below
  till the end the next section.
%
%\end{remark}


%
Let us apply \refeq{primfac}
to the examples in \refsec{examples}.
%%
For $n = 1$,
%there is no irrelevant factor in \refeq{xr1} and
$P_1(\lambda) = B_1^1(\lambda) = A_1(\lambda)
  = \lambda^2 - 2\,\lambda - 4\,R$.
%
%
%
For $n = 2$, since
$B_1^2(\lambda)
=
A_1(\sqrt{\lambda})
A_1(-\sqrt{\lambda})
=
(4R-\lambda)^2 -4\lambda$,
%
we have $P_2(\lambda) = A_2(\lambda) / B_1^2(\lambda) = 4 - 4R - \lambda$.
%
%
For $n = 4$,
%we have
%$16 (R-1)^2 - \lambda
%= (4R - 4 + \sqrt{\lambda})(4R - 4 - \sqrt{\lambda})$
%and
%$
%(16 R^2 + \lambda)^2 - 16 (2R+1)^2 \lambda
%=
%\big[(4R - \sqrt{\lambda})^2  - 4\sqrt{\lambda}\,\big]
%\big[(4R + \sqrt{\lambda})^2  + 4\sqrt{\lambda}\,\big]$.
%Thus,
the last two factors of the polynomial in \refeq{xr4}
can be written as
$B_2^2(\lambda) = A_2(\sqrt{\lambda}) A_2(-\sqrt{\lambda})$,
and
%
$P_4(\lambda) = A_4(\lambda) / B_2^2(\lambda)$
indeed yields the sextic factor.
%$    = 4096 R^6-12288 R^5+ 256 (\lambda + 48) (R^4- R^3)
%    -16(\lambda + 32)(\lambda - 16) R^2 - (\lambda-16)^3$.
%Note, $B_1^4$ is unused for $\mu(4) = 0$.






%\begin{remark}
The proof of \refeq{primfac} and subsequent discussions
  are somewhat technical, so
  the reader may wish to skip the rest of \refsec{logmap}
  on first reading.
%
Since the above derivation works also for complex cycles,
  each complex $R$ root of the polynomial $P_n\Rlam$ traces
  the outline of a bulb in the inverted Mandelbrot set,
  %(the inversion $z_k = -x_k, c = -R$
  %reduces \refeq{logmaps} to the standard Mandelbrot form: $z_{k+1} = c + {z_k}^2$
  %\cite{mandelbrot}),
  as $\lambda=\exp(i\theta)$ goes around the unit circle,
  cf. \cite{stephenson2, stephenson3}. % and \reffig{logbulb}.
%\end{remark}
%





\subsection{\label{sec:degprimfac}Counting cycles}


Below we show \refeq{primfac} based on the fact that
  each zero of the characteristic polynomial $A_n\lam$
  corresponds to a \emph{distinct} $d$-cycle with $d|n$.
%\refeq{primfac} then follows from a simple M\"obius inversion.

%We now prove \refeq{primfac}.
%%
%To do so, we first find the degrees in $\lambda$ of
%  $A_n\lam$ [\refeq{necklace}] and $P_n\lam$ [\refeq{lyndon}].
%By comparing the degrees,
%  we then show that each $P_d\lam$ with $d|n$,
%  after some transformation,
%  contributes a polynomial factor to $A_n\lam$
%  [\refeq{prod}],
%  and the inversion of the relation
%  yields \refeq{primfac}.



%\subsubsection{Number of the square-free cyclic polynomials}

First, to count the number of simple square-free cyclic polynomials,
  we notice a one-to-one mapping
  from the simple square-free cyclic polynomials
  to the binary necklaces,
%
which are defined as nonequivalent binary $0$-$1$ strings.
%
Two strings are equivalent if they differ only by a circular shift.
%
For example, for $n = 3$, there are $2^3 = 8$ binary strings,
but only four necklaces: 000, 001, 011 and 111,
since 010 and 100 are equivalent to 001,
so are 110 and 101 to 011.
%
%The mapping is constructed as follows.
%
Now given a simple square-free cyclic polynomial, if its generator contains $x_k$,
  the $k$th character from the right of the necklace is 1,
  otherwise 0.
For example, the above four necklaces correspond to
  $C_0$, $C_1$, $C_{12}$ and $C_{123}$, respectively.
%For example, if $n = 3$,
% the square-free polynomials for $000$, $100$, $110$ and $111$ are
%  $C_0 = 1$ (generator: 1),
%  $C_1 = x_1 + x_2 + x_3$ (generator: $x_1$),
%  $C_{12} = x_1 x_2 + x_2 x_3 + x_3 x_1$ (generator: $x_1 x_2$)
%  and $C_{123} = x_1 x_2 x_3$ (generator: $x_1 x_2 x_3$), respectively.
Thus,
%
%\begin{theorem}
the number $\NB(n)$ of the square-free cyclic polynomials
  is equal to the number of length-$n$ binary necklaces,
  which is given by \cite{riordan}
\begin{equation}
  %\deg_\lambda A_n\lam =
  \NB(n) = (1/n) \textstyle\sum_{d|n} \varphi(n/d) 2^{\,d},
\label{eq:necklace}
\end{equation}
%
%\label{thm:necklace}
%\end{theorem}
%
where $\varphi(m)$ is Euler's totient function,
  which counts the number of integers from 1 to $m$
  that are coprime to $m$,
  e.g.,
  $\varphi(1) = 1$,
  $\varphi(6) = 2$ for 1 and 5 are coprime to 6 \cite{hardy}.
%
By \refeq{xr},
  $\deg_\lambda A_n\lam = \NB(n)$.
The first few $\NB(n)$ are
2, 3, 4, 6, 8, 14, 20, 36, 60, 108, \dots, % 188, 352, 632, \ldots,
  from $n = 1$.



%\subsubsection{Number of the $n$-cycles}


%\begin{theorem}
Next, we define the $n$-cycle polynomial as
\begin{equation}
  P_n\lam \equiv \textstyle\prod_l \Big[
    \Lambda\big(\vx{n}^{(l)}\big) - \lambda
  \Big],
\notag
\end{equation}
where $\vx{n}^{(l)}$ denotes the $l$th $n$-cycle.
%
Thus, $\deg_\lambda P_n\lam$
  is equal to the number $\LB(n)$ of the $n$-cycles,
  which is given by
  \cite{hao, hao2, lutzky}
\begin{equation}
  %\deg_\lambda P_n\lam =
  \LB(n) = (1/n)\textstyle\sum_{d | n} \mu(n/d) 2^{\,d}.
\label{eq:lyndon}
\end{equation}
%\label{thm:lyndon}
%\end{theorem}
%
%
%\end{proof}
%
The first few $\LB(n)$ are
2, 1, 2, 3, 6, 9, 18, 30, 56, 99, \dots %186, 335, 630, \ldots,
from $n = 1$.
%
%
One can verify that
%Incidentally, $\LB(n)$ is also the number of aperiodic necklaces of length $n$
%(e.g., for $n = 4$,
%  ``$0011$'' is aperiodic, but ``$0101$'' is not, owing to the shorter period 2).
%%
%An $n$-necklace is either aperiodic itself or
%  a repeat of an aperiodic $d$-necklace with $d|n$; so
\begin{equation}
  \NB(n) = \textstyle\sum_{d|n} \LB(d).
\notag
%\label{eq:necklacelyndon}
\end{equation}
%
%Since the left side is the degree in $\lambda$ of $A_n\lam$,
Thus, in order for $A_n\lam$ to accommodate all fixed points of $f^n$,
  each distinct $d$-cycle ($d|n$)
  contributes one and only one zero to $A_n\lam$.
%
%
Comparing the leading coefficient of $A_n\lam$ leads to
%is unity by \refeq{xr}, we must have
%\begin{theorem}
  \begin{equation}
    A_n\lam = \textstyle\prod_{d|n} Q_d^{n/d}\lam,
    \label{eq:prod}
  \end{equation}
where $Q_d^{n/d}(\lambda) \equiv \prod_l
  \big[\Lambda\big(\vx{n}^{(l)}\big) - \lambda\big]$
  with $\vx{n}^{(l)}$ being the $n$ consecutive points
  in the $l$th $d$-cycle,
%
  or the $d$-cycle points $\vx{d}^{(l)}$
  repeated for $c = n/d$ times:
\begin{equation}
  \vx{n}^{(l)}
  =
  \vx{d}^{(l)} \times c
  =
  \{
    x_1^{(l)}, \ldots, x_d^{(l)},
    \; \ldots, \;
    x_1^{(l)}, \ldots, x_d^{(l)}
  \}.
\notag
\end{equation}
%
It follows that
$\Lambda(\vx{n}) = \left[ \Lambda(\vx{d}) \right]^c$.
%
By factoring $\Lambda^c - \lambda$
  in terms of the $c$ complex roots of $\lambda$,
  we get
%
\begin{align}
Q_d^{c}\lam
%  &= \prod_l
%    \Big\{
%      \lambda - \big[\Lambda\big(\vx{d}^{(l)}\big) \big]^{c}
%    \Big\}
%    \notag \\
  &=
    \prod_l
    \left\{
      \prod_{k=1}^c \,
      \left[
          \Lambda\big(\vx{d}^{(l)}\big)
          -
          e^{2k\pi i/c} \lambda^{1/c}
      \right]
    \right\}
    \notag \\
  &=  \prod_{k=1}^c P_d\big(e^{2k\pi i/c} \lambda^{1/c}\big).
\label{eq:QP}
\end{align}
%
%where we have exchanged the order of the two products in the second step.
%
The radical $\lambda^{1/c}$
  does not prevent
  $Q_d^c\lam$ from being a polynomial of $\lambda$,
  thanks to the invariance
  $\lambda \rightarrow e^{2\pi i} \lambda$;
and
  $\deg_\lambda Q_d^c\lam
    = \deg_\lambda P_d\lam = \LB(d)$.


%Therefore, the product $\prod_{cd=n} Q_d^c\lam$
%  can differ from $A_n\lam$ only by
%  a multiple.
%Since $Q_n^1\lam = P_n\lam$,
%  and the coefficient of highest power of $\lambda$ is always
%  unity in $A_n\lam$ \big[see the definition \refeq{xr}\big],
%  we know by induction that the coefficients of the highest power of $\lambda$
%  in all $P_n\lam$ and $Q_d^c\lam$
%  are also unities.
%So the multiple is one, hence \refeq{prod}.




Now, from \refeqs{prod} and (\ref{eq:QP}), we have
%\begin{small}
\begin{align*}
 B_n^m (\lambda)
% & = \prod_{r = 1}^{m} A_n(e^{2l\pi i/m} \lambda^{1/m}) \\
 & = \prod_{k' = 1}^m \prod_{c | n} \prod_{k=1}^c
      P_{n/c}\Big(
        e^{2k\pi i/c}
        \big(
          e^{2 k' \pi i/m} \lambda^{1/m}
        \big)^{1/c}
      \Big) \\
 & = \prod_{c | n} \prod_{k''=1}^{m c}
    P_{n/c}\big(
      e^{2k''\pi i/(m c)} \lambda^{1/(m c)}
      \big) \\
 &
  = \prod_{c | n} Q_{n/c}^{m c}(\lambda)
  = \prod_{d | n} Q_{d}^{m n/d}(\lambda).
\end{align*}
%\end{small}
The M\"obius inversion yields
\begin{equation}
Q_n^m\lam
= \textstyle\prod_{d|n}
  \left[
    B_d^{m n/d}\lam
  \right]^{\mu(n/d)}.
\notag
%\tag{$\ref{eq:primfac}'$}
\end{equation}
\refeq{primfac} is the special case of $m = 1$,
for $Q_n^1(\lambda) = P_n(\lambda)$.









\subsection{\label{sec:origfac}%Intersection of cycles and
  Factors at the onset point}


The $n$-cycle polynomial $P_n(R, \lambda)$
  is generally irreducible.
At the onset point $\lambda = +1$,
  however, $P_n(R, +1)$ can be further factored,
  cf. the paragraph after \refeq{xr4} for the $n = 4$ case.
%
This factorization is due to the intersection with
  a shorter $d$-cycle with $d|n$.
%
The intersection forces the two cycles to share orbit,
  resulting a period-$(n/d)$-tupling
  of the shorter $d$-cycle \cite{hao}.
%
One can show that i) the period tupling happens only
  at the onset of the longer $n$-cycle, where $\lambda = 1$,
%
  and ii) $d$ and $n$ are coprime \cite{blackhurst}.
%
As a result, the $n$-cycle polynomial $P_n(R, \lambda)$
  has to accommodate the $d$-cycle one $P_d(R, \lambda')$ as a factor,
  with $\lambda'$ being a primitive $(n/d)\,$th root of $\lambda = 1$.
%
The simplest example is the first bifurcation point
  at $R = 3/4$ for $d = 1$ and $n = 2$,
  where the fixed point %\refeq{xr1}
  undergoes a period-doubling and becomes the 2-cycle. % \refeq{xr2}.


We can thus distinguish \emph{original} cycles
  from those born out of the above cycle intersection. % period-tupling.
%
Similar to \refeq{QP},
  the unoriginal cycles
  from the shorter $d$-cycles with $d|n$
  contributes a cyclotomic product
  $\prod_{(k, c) = 1}
      P_{d}
        \left(
          R, e^{2k\pi i/c}
        \right)$ to $P_n(R, 1)$,
where $k$ runs through integers from 1 to $c=n/d$
  that are coprime to $c$.
%
This product for cycle intersection is equivalent to a previous
  formula in terms of canonical matrices \cite{blackhurst}.
%
With these contributions removed,
  we obtain the polynomial responsible for the original cycles:
  \begin{equation}
    S_n(R)
    = \frac
    {
      P_n(R, 1)
    }
    {
      \prod_{c d =  n, \; d < n}
      %\left[ \,
        \prod_{(k, c) = 1}
      P_{d}
        \left(
          R, e^{2k\pi i/c}
        \right)
      %\right]
    }.
  \label{eq:origfac}
  \end{equation}


%
%
%
Take the $n = 4$ case as an example
[cf. \refsec{examples}],
we have
$\prod_{k=1,3} P_1(R, e^{k\pi i/2})
%=(-4R-2i-1)(-4R+2i-1)
=(4R+1)^2+4$
(for $c = 4$),
$P_2(R, -1) = 5 - 4R$
(for $c = 2$, the period-doubled 2-cycle).
%
Dividing $P_4(R, 1)$
by the two factors yields
$S_4(R) = 108-(4R-3)^3$.
%  whose only real root $R_a=(3+\sqrt[3]{108})/4$ corresponds to
%  the onset of the original cycle.
%Note that $P_2(R,-1)$ is excluded because it corresponds to the 4-cycle
%  born out of period-doubling the 2-cycle.
%The original and unoriginal 4-cycles are compared
%  in \reffigs{cobweb}(e) and (d).



%We now prove \refthm{origfac}.
%Suppose $n = c \,d$, we have, from \refeq{logmaps},
%\[
%  x_{l+1} - x_{d+l+1} = - (x_l + x_{d+l}) (x_l - x_{d+l}).
%\]
%We apply the equation to $l = 1, \ldots, m$, and the product is
%\[
%  x_{m+1} - x_{m + d+1} =
%  (-1)^{m} \left[ \, \prod_{l=1}^{m} (x_l + x_{d+l}) \right]
%    (x_1 - x_{d+1}).
%\]
%We now set $m$ to $0, d, \ldots, (c-1)\,d$ in this equation,
%  add them together, eliminate $x_1 - x_{d+1}$
%  (which is nonzero in a cycle), and
%\begin{equation}
%  \sum_{c' = 0}^{c-1}
%    (-1)^{c' d} \prod_{l=1}^{c' d} (x_l + x_{d+l})
%   = 0.
%\label{eq:stair}
%\end{equation}
%%
%Note \refeq{stair} holds for every divisor $c$ of $n$ ($c > 1$). We now have
%
%
%
%\begin{theorem}
%An $n$-cycle and a shorter $d$-cycle ($d|n$, $d< n$)
%  intersect only at the onset of the $n$-cycle,
%%
%and $\prod_{k=1}^{d} f'(x_k) = (-2)^d x_1 \dots x_d$ is
%a \emph{primitive} $(n/d)$th root of unity there.
%\label{thm:cbifur}
%\end{theorem}
%
%
%\begin{proof}
%At the intersection of the $n$- and $d$-cycles,
%  $x_l$ repeats itself after $d$ steps, so
%  $x_{d + l} = x_l$;
%and \refeq{stair} becomes,
%%
%\begin{equation}
%  1 + q + \dots + q^{c-1} = 0,
%\label{eq:cbifur}
%\end{equation}
%%
%where $q = (-2)^d \, x_1 \dots x_d$.
%%
%Multiplying \refeq{cbifur} by $q - 1$ yields
%  $1 = q^c = (-2)^n \, x_1 \dots x_n$.
%So the $n$-cycle is at its onset.
%
%
%Further $q$ is a \emph{primitive} $c$th root of unity.
%%
%Suppose the contrary: $q = e^{2k\pi i/c}$ and $(k, c) = g >1$,
%  then by $c_1 \equiv c/g$, $k_1 \equiv k/g$, we have
%\begin{equation}
%  q = e^{2k_1\pi i/ c_1}.
%\label{eq:qd}
%\end{equation}
%Similar to \refeq{cbifur}, we can apply \refeq{stair} with $c \rightarrow g$ and $d \rightarrow d c_1$, and
%\[
%  1 + q_1 + \dots + {q_1}^{g-1} = 0,
%\]
%  where $q_1 = (-2)^{d c_1} \, x_1 \dots x_{dc_1} = q^{c_1}$.
%But by \refeq{qd}, $q^{c_1} = e^{2 k_1\pi i} = 1$,
%and $1 + q_1 + \dots + {q_1}^{g-1} = g > 0$; a contradiction.
%%
%\end{proof}
%
%
%\begin{remark}
%The only real $q$ is $q = -1$ for $c=2$, i.e., a period-doubling.
%%
%On the complex domain, however, we can have a $c$-fold branching with $c>2$,
%  which corresponds to a contact points between ``bulbs''
%  in the inverted Mandelbrot set.
%\end{remark}
%
%By \refthm{cbifur}, $P_n\Rlam$ at the onset point
%includes $P_d(R, e^{2 k \pi i/c} \lambda^{1/c})$ for every
%  possible combination of $k$ and $c$, such that
%  $(k, c) = 1$, $c|n$, and $c > 1$.
%
%Dividing the factors from $P_n\Rlam$ yields \refthm{origfac}.






%\subsection{\label{sec:degR}Degrees in $R$}
%
%
%
%In the numerical implementation
%  (cf. \refsec{end}),
%  the degrees in $R$ of the polynomials are needed
%  for evaluating the determinant by polynomial interpolation.
%We now show that
%\begin{equation}
%\deg_R A_n\Rlam = \textstyle\sum_{d|n} \varphi(n/d) 2^{d-1},
%\label{eq:degRA}
%\end{equation}
%%
%The first few values are
%1, 3, 6, 12, 20, 42, 70, 144, 270, 540, \dots, from $n =1$.%1034, 2112, 4108, \dots,
%
%From \refeq{xr},
%  every term in the determinant $|\vct A|$
%  can be written as
%  $\prod_{p} A_{p,q}$,
%  in which, $p$ runs through all cyclic polynomials,
%  and $\{q = q(p)\}$ is a permutation of $\{p\}$.
%Thus, $\deg_R |\vct A|$ is
%  reached by the term that maximizes the product.
%Recall that $A_{p,q}$ results from reducing
%  $\Lambda(\vx{n}) \, C_p(\vx{n})$ to $C_q(\vx{n})$
%  by a sequence of substitutions ${x_k}^2 = R-x_{k+1}$.
%It follows that to increase $\deg_R$ by 1,
%  $\deg_x$ is decreased by at least 2.
%Accordingly,
%  $\deg_R A_{p,q} \le \deg_x(\Lambda \, C_p) - \deg_x C_q$.
%Now running $p$ and $q$ through the $\NB$ cyclic polynomials
%  and using $\deg_x \Lambda(\vx{n}) = n$
%  yields
%\[
%  \deg_R |\vct A | = \deg_R \textstyle\prod_{p} A_{p,q} \le n \, \NB(n).
%\]
%%
%The equality is achieved, if each $q$ is the complement $\bar p$ of $p$,
%  that is, if $C_p$ is generated from $\prod_k x_k^{e_k}$,
%  $C_{\bar p}$ can be generated from $\prod_k x_k^{1-e_k}$
%%  (which may be the same polynomial as $C_p$).
%For example, for $n=4$,
%  $C_1 = x_1 + \cdots + x_4$
%  is the complement of
%  $C_{123} = x_2 x_3 x_4 + \cdots + x_1 x_2 x_3$,
%  while $C_{12}$ is its own complement.
%Finally, we reach \refeq{degRA} by \refeq{necklace}.
%
%
%Now, taking the degrees in $R$ of \refeq{primfac} and \refeq{origfac}
%  yields two known formulas
%\cite{mira, stephenson1, blackhurst}:
%%
%\begin{subequations}
%\begin{align*}
%\deg_R P_n\Rlam &\equiv \beta(n)
%  = \sum_{d|n} \mu\Big(\frac{n}{d}\Big) 2^{d-1}, \\
%\deg S_n(R) &= \beta(n)
%  - \sum_{d | n, d < n} \beta(d) \, \varphi\Big(\frac{n}{d}\Big).
%\end{align*}
%\label{eq:degR}
%\end{subequations}
%%
%The first few $\deg_R P_n\Rlam$ are
%  1, 1, 3, 6, 15, 27, 63, 120, 252, 495, \dots, %1023, 2010, 4095, \dots,
%and $\deg S_n(R) = $
%  1, 0, 1, 3, 11, 20, 57, 108, 240, 472, \dots, % 1013, 1959, 4083, \dots,
%from $n = 1$.
%



%\begin{table}[h]
%  %\small
%  \tbl{
%  Smallest positive $r$ at the onset and bifurcation points
%  of the $n$-cycles of the logistic map.
%  }
%{
%\begin{tabular}{llll}
%%\begin{tabularx}{\linewidth}{
%%  >{\hsize=0.5\hsize\centering\arraybackslash}X
%%  >{\hsize=1.6\hsize}X
%%  >{\hsize=1.6\hsize}X
%%  >{\hsize=0.3\hsize\raggedright\arraybackslash}X
%%}
%\hline
%  $n^\dagger$
%& Onset$^\ddagger$
%& Bifurcation$^\ddagger$
%& \#$^*$ \\
%\hline
%$1$     & $1.0000000000_1$      &  $3.0000000000_1$       & 1   \\
%$2'$    & $3.0000000000_1$      &  $3.4494897428_1$       & 1   \\
%$3$     & $3.8284271247_1$      &  $3.8414990075_3$       & 1   \\
%$4$     & $3.9601018827_3$      &  $3.9607686524_6$       & 1   \\
%$4''$   & $3.4494897428_1$      &  $3.5440903596_6$       & 1   \\
%$5$     & $3.7381723753_{11}$   &  $3.7411207566_{15}$    & 3   \\
%$6$     & $3.6265531617_{20}$   &  $3.6303887000_{27}$    & 4   \\
%$6'$    & $3.8414990075_{3}$    &  $3.8476106612_{27}$    & 1   \\
%$7$     & $3.7016407642_{57}$   &  $3.7021549282_{63}$    & 9   \\
%$8$     & $3.6621089132_{108}$  &  $3.6624407072_{120}$   & 14  \\
%$8'$    & $3.9607686524_{6}$    &  $3.9610986335_{120}$   & 1   \\
%$8'''$  & $3.5440903596_{6}$    &  $3.5644072661_{120}$   & 1   \\
%$9$     & $3.6871968733_{240}$  &  $3.6872742105_{252}$   & 28  \\
%$10$    & $3.6052080669_{472}$  &  $3.6059169323_{495}$   & 48  \\
%$10'$   & $3.7411207566_{15}$   &  $3.7425706462_{495}$   & 3   \\
%$11$    & $3.6817160194_{1013}$ &  $3.6817266457_{1023}$  & 93  \\
%$12$    & $3.5820230011_{1959}$ &  $3.5828117795_{2010}$  & 165 \\
%$12'$   & $3.6303887000_{27}$   &  $3.6321857392_{2010}$  & 4   \\
%$12''$  & $3.8476106612_{27}$   &  $3.8490363152_{2010}$  & 1   \\
%$13$    & $3.6797024578_{4083}$ &  $3.6797038498_{4095}$  & 315 \\
%$14$    & $3.5972109838_{8052}$ &  $3.5973201609_{8127}$  & 576 \\
%$14'$   & $3.7021549282_{63}$   &  $3.7024116236_{8127}$  & 9   \\
%\hline
%\multicolumn{4}{p{\linewidth}}{
%$^\dagger$
%  $\,'$, $\,''$, or $\,'''$ means
%    that the cycle is undergoing
%    the first, second, or third successive period-doubling, respectively.
%} \\
%\multicolumn{4}{p{\linewidth}}{
%$^\ddagger$
%  The subscripts are the degrees of the corresponding minimal polynomial
%    of $R = r(r-2)/4$.
%} \\
%\multicolumn{4}{p{\linewidth}}{
%$^*$
%  The number of similar cycles.
%} \\
%\hline
%%\end{tabularx}
%\end{tabular}
%\label{tab:rval}
%}
%\end{table}





\subsection{\label{sec:end}Implementation}


%Let us summarize the algorithm.
%%
%First, we list \refeqs{xcp}  with $\Lambda(\vx{n}) = \prod_{k=1}^n f'(x_k)$.
%This step populates elements of the matrix $\vct T$.
%%
%The characteristic equation $A_n(\lambda) = \big|\lambda \, \vct I - \vct T\big| = 0$,
%with $\lambda$ being $+1$ and $-1$, then gives
%the polynomials at onset and bifurcation points, respectively.
%%
%The contributions from shorter $d$-cycles
%  can be removed by
%  % computing the characteristic equations for all divisors $d$ and applying
%  \refeq{primfac}.
%%

The most time-consuming step of the algorithm in \refsec{algo}
  is the evaluation of the determinant $A_n(R, \lambda = \pm1)$
  in the characteristic equation (\ref{eq:xr}).
%
In implementation,
we found it helpful to use polynomial interpolation
by first evaluating $A_n\Rlam$ at $\deg_R A_n + 1$ values of $R$,
e.g., $R = 0, \pm1, \pm2,\ldots$, then piecing them together
into a polynomial of $R$.
%
According to \refeq{primfac}, we only need $\deg_R P_n + 1$ values
  to directly compute $P_n\Rlam$.
%
%
Here,
$\deg_R A_n = n \NB(n)/2 = \sum_{d|n} \varphi(n/d) 2^{d-1}$
and $\deg_R P_n = n \LB(n)/2 = \sum_{d|n} \mu(n/d) 2^{d-1}$
  \cite{mira, stephenson1}.
%
%The strategy also allows a trivial parallelization.
%
%\begin{figure}[h]
%  \begin{center}
%  \begin{minipage}{\linewidth}
%        \includegraphics[width=\linewidth]{T14b.png}
%  \end{minipage}%
%  \end{center}
%  \caption{\label{fig:T14b}
%  Coefficients of the 14-cycle bifurcation polynomial in two columns (left to right).
%  The $k$th line from the top is the coefficient before $(4R)^k$ in base-256.
%  The digits are shown by the grayscale from white (0) to black (255).
%  The leading bar before each line is the sign (black: $-$, white: $+$).
%  }
%\end{figure}


We coded the algorithm into a Mathematica program,
  and computed the polynomials for $n$ up to 14.
The data are saved on the website http://logperiod.appspot.com.
%
%The bifurcation polynomial of the 14-cycle in shown in \reffig{T14b}.
%
%The exact polynomials of these maps are generally too large to print on paper,
%  e.g., the bifurcation polynomial of the 14-cycle
%    takes about 27 megabytes.
%
%The representative $r$ values are listed in \reftab{rval}.
%
%Compared with other approaches \cite{hitzl, gordon, bailey2, kk1, hao, hao2}
%it appeared to be simpler and more transparent,
%as it only requires the computation of a determinant.
%
As shown in \reffig{logtime}, the algorithm
  outperformed the resultant \cite{burm} and
  Gr\"obner basis \cite{kk1} methods.
%  by at least two orders of magnitudes.
For the implementation of the resultant method on Magma, we have included a version
  that uses the polynomial interpolation technique
  similar to the one described in the previous paragraph.
This version performed slightly better than the purely symbolic version.
The small oscillation in the running time is due to that the resultant contains
  only the $n$th power of the $n$-cycle polynomial $P_n\Rlam$,
thus, to recover the sign of $P_n\Rlam$ for an even $n$,
  the number of evaluation points need to be doubled.
%
This is unsurprising,
  because the resultant of $f^n(x)-x$ and $\frac{d}{dx}f^n(x)-\lambda$
  is the determinant of the Sylvester matrix, which has
  $2^{n+1}-1$ rows and columns
  [as $\deg_x f^n(x) = 2^n$],
  while the coefficient matrix in \refeq{xr}
  has only $\NB \approx 2^n/n$ [cf. \refeq{necklace}] rows and columns.
%Thus, the current algorithm is expected to be faster.



\begin{figure}[h]
  \begin{center}
  \begin{minipage}{\linewidth}
        \includegraphics[angle=-90, width=\linewidth]{logtime.pdf}
  \end{minipage}%
  \end{center}
  \caption{\label{fig:logtime}
  Time to compute the $n$-cycle bifurcation polynomials,
    measured from a single core of
    Intel\textsuperscript{\textregistered} 2.5GHz Dual-Core CPU
  (Mathematica version: 8, Magma version: 2.18-4).
  }
\end{figure}
%










\section{\label{sec:henon}H\'enon map}




The H\'enon map \cite{henon}
\begin{equation}
  X_{k+1} = 1 + Y_k - a \, {X_k}^2, \quad
  Y_{k+1} = b \, X_k,
%\label{eq:henon}
\notag
\end{equation}
is a two-dimensional extension of the logistic map.
%
By
  $x_k \equiv a \, X_k,
   y_k \equiv a \, Y_k$
\cite{huang},
we have
\begin{equation}
  x_{k+1} = a + y_k - {x_k}^2, \quad
  y_{k+1} = b \, x_k.
%\tag{$\ref{eq:henon}'$}
\label{eq:henons}
\end{equation}
%
%Since neither $a$ nor $b$ is changed during the transformation,
%\refeq{henon} and \refeq{henons} share
%the same onset and bifurcation points
%in terms of $a$ and $b$.
%
With $b \rightarrow 0$ and $a \rightarrow R$,
we recover the logistic map \refeq{logmaps}.
%
%
We may further eliminate $y_k$ using the second equation of \refeq{henons},
  and work with the cyclic polynomials of only $x_k$.
The square-free reduction is now
  ${x_k}^2 \rightarrow a + b \, x_{k-1} - x_{k+1}$.



The counterpart of \refeq{der}
  can be found from the Jacobian of the $n$th iterate of \refeq{henons}
\[
  \vct J_n
  \equiv
  \left(
    \begin{array}{ccc}
      \partial x_{n+1}/\partial x_1 & \partial x_{n+1}/\partial y_1 \\
      \partial y_{n+1}/\partial x_1 & \partial y_{n+1}/\partial y_1 \\
    \end{array}
  \right),
\]
and the magnitude of both eigenvalues of the matrix must be less than 1.
By the chain rule, we have
  $\vct J_n (x_1) = \vct J(x_n) \cdots \vct J(x_1)$,
  where
$\vct J(x_k)
  =
  \left(
    \begin{array}{cc}
      -2 x_k & 1 \\
      b & 0
    \end{array}
  \right)$;
and the eigenvalues $\lambda$ of $\vct J_n$ satisfy
%
\begin{equation}
\big|\, \lambda \, \vct I - \vct J_n(x_1) \,\big|
= \lambda^2 - \Theta(\vx{n}) \,\lambda +(-b)^n = 0,
\notag
%\tag{$\ref{eq:der}'$}
%\label{eq:henonder}
\end{equation}
%
where $\Theta(\vx{n})$ and $(-b)^n$
  are the trace and determinant of $\vct J_n(x_1)$,
  respectively \cite{huang, hitzl}.
%The onset and bifurcation points
%  correspond to $\lambda = +1$ and $-1$, respectively.
%
%
Using the matrix identity $\Tr( \vct A \vct B) = \Tr(\vct B \vct A)$,
  we have
    $\Tr \, \vct J_n(x_1) = \cdots = \Tr \, \vct J_n(x_n)$,
  i.e., $\Theta(\vx{n})$ is a cyclic polynomial. % of $\vx{n} = \{x_1, \dots, x_n \}$.
%e.g., %$\Theta(\vx{n}) = 4 x_1 x_2 + 2 \, b$ for $n = 2$ and
%$\Theta(\vx{n}) = -8 x_1 x_2 x_3 - 2 \, b (x_1+ x_2+ x_3)$ for $n = 3$.
%
  Thus, we can expand $\Theta(\vx{n}) \, C_p(\vx{n})$
  in terms of square-free cyclic polynomials,
  and then replace the eigenvalue of the coefficient matrix
  by $\lambda + (-b)^n/\lambda$
  in the characteristic equation.
%
%
%
The onset ($\lambda = 1$) and bifurcation ($\lambda = -1$)
  polynomials of $a$ and $b$
  for $n$ up to 9 are saved on the website in \refsec{end}.











\section{\label{sec:cubic}Antisymmetric cubic map}



Consider a cubic version of the logistic map \refeq{logmap}:
\begin{equation}
  z_{k+1} = r \, z_k \, (1 - {z_k}^2).
  \notag
\end{equation}
%
By the transformation $x_k = \sqrt{r} z_k$,
%
we obtain a variant
\cite{strogatz}
%
\begin{equation}
  x_{k + 1} = f_r(x_k) = r \, x_k - {x_k}^3,
\label{eq:cubic}
\end{equation}
%
which is more suitable for algebraic manipulation.
%
This transformation, however, introduces an artifact:
  a real $\vz{n}$ cycle at a negative $r$
  corresponds to an imaginary $\vx{n}$ cycle.
%
We ignore the artifact below,
for it does not affect the boundary polynomials.
%
%For $n>1$, half of the $L^*_e(n)$ cycles
%  with negative $r$ are imaginary.
%They can, however, be made real by
%the transformed map
%shares the same boundary polynomials of $r$ with \refeq{cubic}.
%
Another related cubic map
  $q_{k+1} = a {q_k}^3 + (1-a) q_k$ \cite{may2, brown3, brown4}
  can be studied through $q_k = x_k/\sqrt{r-1}, a = 1-r$.



The solution is similar to the logistic case.
%
The new replacement rule
\begin{equation}
  {x_k}^3 \rightarrow r \, x_k - x_{k+1},
\label{eq:cubreplace}
\end{equation}
however, eliminates only cubes instead of squares of $x_k$.
%
Thus, we must extend the basis of simple square-free cyclic polynomials
  to that of simple cube-free ones
  in listing equations \eqref{eq:xcp}.
For example,
  $C_{112} = {x_1}^2 x_2 + \dots + {x_n}^2 x_1$
  is now included in the basis
  (but
  $C_{1112} = {x_1}^3 x_2 + \dots + {x_n}^3 x_1$
  is still not).
The new $\Lambda_r(\vx{n})$ is
  $\prod_{k=1}^n f'_r(x_k) = \prod_{k=1}^n (r - 3 {x_k}^2)$.
%
%
The polynomials of $r$
  at the onset and bifurcation points
%for some small $n$ are shown in \reftab{cubpolygen} (general $\lambda$)
%and \reftab{cubpoly} ($\lambda = \pm1$);
for $n$ up to 9
%we have saved the data
  are saved on the website in \refsec{end}.
%
%The representative $r$ values are listed in \reftab{crval}.
%


%\begin{table}[h]%\footnotesize
%%\caption{
%\tbl{
%  Smallest positive $r$ at the onset and bifurcation points
%  of the $n$-cycles of the antisymmetric cubic map.
%  }
%%\begin{center}
%{
%\begin{tabular}{l l l l}
%%\begin{tabularx}{\textwidth}{
%%  >{\hsize=0.5\hsize\centering\arraybackslash}X
%%  >{\hsize=1.6\hsize}X
%%  >{\hsize=1.6\hsize}X
%%  >{\hsize=0.3\hsize\raggedright\arraybackslash}X
%%}
%%\begin{tabular*}{\textwidth}{l l l l c | c l l l l}
%\hline
%  $n^\dagger$
%& Onset$^\ddagger$
%& Bifurcation$^\ddagger$
%& \#$^*$ \\
%\hline
%$1$     & $1.0000000000_1$      &  $2.0000000000_1$       & 2   \\
%$2'$    & $2.0000000000_1$      &  $2.2360679775_2$       & 2   \\
%$3$     & $2.4504409645_{8}$    &  $2.4608286739_{12}$    & 4   \\
%$4$     & $2.5478350393_{22}$   &  $2.5488312193_{32}$    & 8   \\
%$4''$   & $2.2360679775_{2}$    &  $2.2880317545_{32}$    & 2   \\
%$5$     & $2.3939250274_{112}$  &  $2.3957922744_{120}$   & 24  \\
%$6$     & $2.3334877526_{304}$  &  $2.3355337580_{336}$   & 56  \\
%$6'$    & $2.4608286739_{12}$   &  $2.4657090579_{336}$   & 4   \\
%$7$     & $2.3729872678_{1080}$ &  $2.3732727868_{1092}$  & 156 \\
%$8$     & $2.3525990555_{3108}$ &  $2.3527637793_{3200}$  & 400 \\
%$8'$    & $2.5488312193_{32}$   &  $2.5493247379_{3200}$  & 8   \\
%$8'''$  & $2.2880317545_{32}$   &  $2.2992279397_{3200}$  & 2   \\
%\hline
%\multicolumn{4}{p{\linewidth}}{
%$^\dagger$
%  $\,'$, $\,''$, or $\,'''$ means that the cycle is undergoing
%    the first, second, or third successive period-doubling, respectively.
%} \\
%\multicolumn{4}{p{\linewidth}}{
%$^\ddagger$
%  The subscripts are the degrees of the corresponding minimal polynomial.
%} \\
%\multicolumn{4}{p{\linewidth}}{
%$^*$
%  The number of similar cycles
%    (for $n > 1$, only half of them have positive $r$).
%} \\
%\hline
%\end{tabular}
%%\end{tabularx}
%%\end{tabular*}
%%\end{center}
%\label{tab:crval}
%}
%\end{table}


The enlarged basis significantly increases the dimension of
  the coefficient matrix in \refeq{xr}, hence the computation cost.
%
Below we discuss two techniques to alleviate the problem.
%by exploiting the certain properties of the map.
%First the antisymmetry: $f_r(-x) = -f_r(x)$
%
First, we may include only about half of the simple cyclic polynomials,
  that is, those of \emph{even} degrees in $\vx{n}$, in the basis.
%
This is because \refeq{cubreplace} %contains only linear and cubic terms,
  never reduces even-degree cyclic polynomials to odd-degree ones,
  or the other way around.
%
The characteristic polynomial obtained
  from the basis of even-degree cyclic polynomials
  is almost the same as that
  from the basis of odd-degree cyclic polynomials,
except that the former admits several symmetric cycles
  that are absent in the latter.



We define a \emph{half-cycle} as a solution of $x_{n+1} = -x_1$,
  with $n$ being the smallest positive integer that satisfies
  the relation.
  %see \reffig{halfcycle} for examples.
%
Once iterated, an $n$-half-cycle turns into a symmetric $2n$-cycle:
$x_1, \ldots, x_{n}, -x_1, \ldots, -x_{n}, x_1$ \cite{hao}.
%
In a symmetric $2n$-cycle,
  an odd-degree cyclic polynomial formed by the $2n$ points
  is always zero,
  e.g., $C_1 = x_1 + \dots + x_n - x_1 - \dots - x_n = 0$.
%
Thus, the symmetric cycles may escape the characteristic equation
  from the basis of odd-degree cyclic polynomials,
  resulting a different $n$-cycle polynomial from
  the one derived from the basis of even-degree cyclic polynomials.


To be more precise, we first define the $n$-half-cycle polynomial $H_n\rlam$
  in analogous to $P_n\rlam$.
%
It is convenient to exclude the contribution
  from the trivial fixed point at $x = 0$
  by defining the \emph{nontrivial} $n$-cycle polynomial
  $P^*_n\rlam \equiv P_n\rlam/(r-\lambda)^{\delta_{n,1}}$,
  and %the nontrivial
  $n$-half-cycle polynomial
  $H^*_n\rlam \equiv H_n\rlam/(r-\lambda)^{\delta_{n,1}}$.
%
%Since a repeated $n$-half-cycle is a $2n$-cycle,
Dividing the $2n$-cycle polynomial
  $P^{*}_{2n}(r, \lambda)$
  by
  $H^{*}_{n}\big(r, \sqrt \lambda\big)
   H^{*}_{n}\big(r, -\sqrt \lambda\big)$
%
yields the \emph{asymmetric} $n$-cycle polynomial:
\begin{equation}
\tilde P^*_n\rlam \equiv
  %\left.
  \frac{
    P^*_n\rlam
  }
  %\middle/
  {
    \left[
    \prod_{\sigma = \pm 1}
      H^*_{n/2}(r, \sigma \sqrt\lambda)
%      H^*_{n/2}(r, -\sqrt\lambda)
    \right]^{\even(n)}
  },
  %\right.
\label{eq:Pasym}
\end{equation}
%]
where $\even(n) \equiv 1 - n\bmod 2$.
%
One can verify that $\tilde P^*_n\rlam$ coincides with the $n$-cycle polynomial
  derived from the basis of odd-degree cyclic polynomials.
%



We now calculate the degrees of the various polynomials
  to facilitate the polynomial interpolation
  mentioned in \refsec{end}.
The number of even-degree cube-free cyclic polynomials
is equal to that of \emph{even ternary} necklaces,
in which each bead represents a number 0, 1, or 2,
  with the total being even.
This number is given by
%
\begin{equation}
  N_e(n) = 1 +
    \frac{1}{2n} \sum_{c d = n} \varphi(c)
    \left[ 2 - \odd(c) \, \right]
    (3^d-1),
\notag
%\label{eq:cubnecklace}
\end{equation}
where
$\odd(c) \equiv c \bmod 2$.
%$N_e(n) =$ 2, 4, 6, 14, 26, 68, 158, 424, \dots, from $n = 1$.
%
%This is roughly half of the number of all (even and odd) ternary necklaces:
%$N_3(n) = \frac{1}{n} \sum_{d|n} \varphi(n/d) 3^d \approx 2 N_e$.
%
%
%Since $N_e(n)$ is the degree in $\lambda$ of $A_n\rlam$,
%  half of the cyclic polynomials and their associated equations are spared.
%
The degree $L_e(n)$ of the $n$-cycle polynomial $P_n\rlam$
  can be obtained from
  $N_e(n) = \sum_{d|n} L_e(d)$,
and
%
%
\begin{equation}
  L_e(n) = \delta_{n, 1} +
    \frac{1}{2n} \sum_{cd = n} \mu(c) \odd(c) \, (3^d-1),
\notag
%\label{eq:cublyndon}
\end{equation}
%
%
%The first few values are 2, 2, 4, 10, 24, 60, 156, 410, \ldots, starting from $n = 1$.
%
which is also the number of the $n$-cycles \cite{zeng},
%
and
  $L^*_e(n) \equiv \deg_\lambda P^*_n\rlam = L_e(n) - \delta_{n,1}$.
%
%For $n>1$, half of the $L^*_e(n)$ cycles
%  with negative $r$ are imaginary.
%They can, however, be made real by
%$x_k = \sqrt{r} z_k$;
%the transformed map
%$z_{k+1} = r \, z_k \, (1 - {z_k}^2)$
%shares the same boundary polynomials of $r$ with \refeq{cubic}.
%
%
One can further show that the degrees in $r$ are
  $\deg_r A_n\rlam = n N_e(n)$
  and
  $\deg_r P^{(*)}_n\rlam = n L^{(*)}_e(n)$.


To find the degrees of $\tilde P^*_n\rlam$ and $H^*_n\rlam$,
  we consider a quadratic transformation $u_k = {x_k}^2/r$,
  which converts \refeq{cubic} to
\begin{equation}
  u_{k+1} = F(u_k) = r^2 u_k (1 - u_k)^2
\label{eq:uk}
\end{equation}
%
Since the new variable $u_k$ discards the sign of $x_k$,
  the $n$-cycles of the new map \refeq{uk} encompass
  both
  asymmetric $n$-cycles
  and
  $n$-half-cycles
  of the original map \refeq{cubic}.
%
We now note $d u_{n+1} / d u_1 = \sigma d x_{n+1} / d x_1$,
  with $\sigma \equiv x_{n+1}/x_1$ equal to $+1$ and $-1$
  in the $n$-cycles
  and
  $n$-half-cycles of \refeq{cubic}, respectively.
%
It follows that the nontrivial
  % (excluding the fixed point $u_k = 0$)
  $n$-cycle polynomial of \refeq{uk},
  whose degree in $\lambda$ is $(1/n)\sum_{cd = n} \mu(c) (3^d - 1)$,
  %which is an even polynomial of $r$,
  can be written as
  $\tilde P^*_n\rlam H^*_n(r, -\lambda)$,
  up to a sign.
%
From this and \refeq{Pasym}, we find that
  $\deg_\lambda \tilde P^*\rlam$ is equal to
%
\begin{align*}
  L_o(n) &= %\delta_{n, 1} +
      \frac{1}{2n} \sum_{cd = n}
    \mu(c) \left[2 - \odd(c)\right] (3^d-1),
\end{align*}
and $\deg_\lambda H^*_n\rlam = \deg_\lambda P^*_n\rlam = L^*_e(n)$.
%with
%$L^*_e(n) = L_o(n) + \even(n) L^*_e(n/2)$.
%



We now turn to the second technique,
  which reduces the computational cost
  by half for an even $n$.
%
If a sequence $x_k$ satisfies \refeq{cubic},
  then $y_k \equiv (-)^k i x_k$
  satisfies $y_{k+1} = f_{-r}(y_k)$.
%
Now for an even $n$,
  an asymmetric $n$-cycle
  $\vx{n} = \{x_1, x_2, \dots, x_n, x_1\}$
  of $f_r$
  can be isomorphically mapped to
  an asymmetric $n$-cycle
  $\vy{n} = \{-i x_1, i x_2, \dots, i x_n, - i x_1\}$
  of $f_{-r}$,
  with $\Lambda_r(\vx{n}) = \Lambda_{-r}(\vy{n})$.
%
So $P^*_n(r, \lambda) = P^*_n(-r, \lambda)$,
  i.e., the asymmetric $n$-cycle polynomial is an even polynomial of $r$
  if $n$ is even.
%
Similarly, the nontrivial $n$-half-cycle polynomial $H^*_n\rlam$
  also has an even degree in $r$ if $n$ is even.
%
Further, $\deg_r P^*_{n}\rlam$ is also even if $4|n$
  by \refeq{Pasym},
  but not so if $n/2$ is odd,
  because a symmetric $n$-cycle % or an $(n/2)$-half-cycle,
  $\vx{n} = \{x_1, \dots, x_{n/2}, -x_1, \dots, -x_{n/2}, x_1\}$
  would be mapped to an $(n/2)$-cycle
  $\vy{n} = \{-i x_1, \dots, - i x_{n/2}, -i x_1, \dots, -i x_{n/2}, -i x_1\}$,
  breaking the isomorphism.
%So, for an odd $m = n/2$, we only have
%  $H^*_m(r, \lambda) = P^*_{m}(-r, -\lambda)$.
%
Thus, the above polynomials need to be evaluated only
  at the positive half of the $r$ values
  in the polynomial interpolation.
%



\nonumsection{Acknowledgments} \noindent
It is a pleasure to thank T. Gilbert, Y. Mei and the referees
  for helpful communications and suggestions.
Computing time on the Shared University Grid at Rice University,
  funded by NSF under Grant EIA-0216467, is gratefully acknowledged.

%\appendix{}

% ws-ijbc dislikes the following
%\appendix


%\appendixpage



%\appendix{\label{apd:per4}Derivation of the 4-cycles}
%
%
%
%The polynomials for the 4-cycles has a compact derivation.
%We first list the explicit equations:
%%
%%
%%
%\begin{subequations}
%\label{eq:x4}
%\begin{align}
%  x_2 &= R - {x_1}^2, \\
%  x_3 &= R - {x_2}^2, \\
%  x_4 &= R - {x_3}^2, \\
%  x_1 &= R - {x_4}^2,
%\end{align}
%\end{subequations}
%%
%$\big[\mathrm{\refeqsub{x4}{a}} - \mathrm{\refeqsub{x4}{c}}\big]
%\times
%\big[\mathrm{\refeqsub{x4}{b}} - \mathrm{\refeqsub{x4}{d}}\big]$
%yields $1 + (x_1 + x_3) (x_2 + x_4) = 0$,
%since $x_1 \ne x_3, x_2 \ne x_4$.
%%
%Hence, with
%$y_1 \equiv x_1 + x_3$, $y_2 \equiv x_2 + x_4$,
%$z \equiv y_1 + y_2$,
%we have
%%
%%
%%
%\begin{subequations}
%\begin{align}
%y_1 y_2           &= -1, \\
%{y_1}^2 + {y_2}^2 &= z^2 - 2 \, y_1 y_2 = z^2 + 2, \\
%{y_1}^3 + {y_2}^3 &= z^3 - 3 \, y_1 y_2 \, z = z^3 + 3 z.
%\end{align}
%\label{eq:ypow4}
%\end{subequations}
%%
%%
%Multiplying \refeqsub{x4}{a} by $x_1$ or $x_3$,
%then summing over cyclic versions yields
%\begin{subequations}
%\begin{align}
%y_1 y_2 &= R z - \big[({x_1}^3 + {x_3}^3) + ({x_2}^3 + {x_4}^3)\big],\\
%y_1 y_2 &= R z - \big[x_1 x_3 \, y_1 + x_2 x_4 \, y_2\big].
%\end{align}
%\label{eq:p4q}
%\end{subequations}
%%
%From
%$\mathrm{\refeqsub{p4q}{a}} + 3 \times \mathrm{\refeqsub{p4q}{b}}$,
%we have
%$4 \, y_1 y_2 = 4 R z - ({y_1}^3 + {y_2}^3)$,
%%
%and by \refeqs{ypow4},
%\begin{equation}
%  z^3 - (4 R - 3) z - 4 = 0.
%  \label{eq:xr4s}
%\end{equation}
%%
%%
%Since
%$2 x_1 x_3 = {y_1}^2 - ({x_1}^2 + {x_3}^2)
%= {y_1}^2 - 2 R + y_2$,
%and $2 x_2 x_4 = {y_2}^2 - 2 R + y_1$,
%%
%%Using \refeqs{ypow4} to simplify the right hand side, we find
%\begin{equation}
%  X \equiv x_1 x_2 x_3 x_4 = \tfrac{1}{2} R z(1 - z) + (R^2 - R + 1),
%\label{eq:der4}
%\end{equation}
%where we have used \refeqs{ypow4} and \req{xr4s} to simplify the result.
%%$X=1/16$ and $-1/16$ at the onset and bifurcation points, respectively.
%%
%Dividing the polynomial in \refeq{xr4s} by that in \refeq{der4}
%yields $z = (R^2-3R-X+1)/(R^2-R+X-1)$,
%and plugging it back to \refeq{der4} gives
%   $R^6 -3 R^5
%  + (3 + X) (R^4 - R^3)
%  + (1 - X) (2 + X) R^2
%  + (1 - X)^3 = 0$,
%which is the same as the sextic factor of \refeq{xr4}
%  with $X = \lambda/16$.
%
%








%\section{\label{apd:cublyndon}Proof of \refthm{cublyndon}}












\begin{thebibliography}{50}


\bibitem[May(1976)]{may}
  May, R. H. [1976]
  ``Simple mathematical models with very complicated dynamics,''
  {\it Nature}
  \textbf{261},
  459--467.

\bibitem[Strogatz(1994)]{strogatz}
  Strogatz, S. H. [1994]
  {\it Nonlinear Dynamics and Chaos}
  (Addison-Wesley. Reading, MA).

\bibitem[Hao(1989)]{hao}
  Hao, B.-L. [1989]
  {\it Elementary Symbolic Dynamics and Chaos in  Dissipative Systems}
  (World Scientific, Singapore).

\bibitem[Brown(1981)]{brown1}
  Brown, A. [1981]
  ``Equations for periodic solutions of a logistic difference equation,''
  {\it J. Austral. Math. Soc. Ser. B}
  \textbf{23},
  78--94
  (1981).

\bibitem[Brown(1982)]{brown2}
  Brown, A. [1982]
  ``Solutions of period seven for a logistic difference equation,''
  {\it Bull. Austral. Math. Soc.}
  \textbf{26},
  263--284.

\bibitem[Stephenson(1991)]{stephenson1}
  Stephenson, J. [1991]
  ``Formulae for cycles in the Mandelbrot set,''
  {\it Physica A}
  \textbf{177},
  416--420.

\bibitem[Stephenson(1992a)]{stephenson2}
  Stephenson, J. [1992a]
  ``Formulae for cycles in the Mandelbrot set II,''
  {\it Physica A}
  \textbf{190},
  104--116.

\bibitem[Stephenson(1992b)]{stephenson3}
  Stephenson, J. [1992b]
  ``Formulae for cycles in the Mandelbrot set III,''
  {\it Physica A}
  \textbf{190},
  117--129.

\bibitem[Saha \& Strogatz(1995)]{saha}
  Saha, P. \& Strogatz, S. H. [1995]
  ``The birth of period three,''
  {\it Math. Mag.}
  \textbf{68},
  42--47.

\bibitem[Bechhoefer(1996)]{bechhoefer}
  Bechhoefer, J. [1996]
  ``The birth of period 3, revisited,''
  {\it Math. Mag.}
  \textbf{69},
  115--118.

\bibitem[Gordon(1996)]{gordon}
  Gordon, W. B. [1996]
  ``Period three trajectories of the logistic map,''
  {\it Math. Mag.}
  \textbf{69},
  118--120.

\bibitem[Burm \& Fishback(2001)]{burm}
  Burm, J. \& Fishback, P. [2001]
  ``Period-3 orbits via Sylvester's theorem and resultants,''
  {\it Math. Mag.}
  \textbf{74},
  47--51.

\bibitem[Zhang(2010)]{zhang}
  Zhang, C. [2010]
  ``Period three begins,''
  {\it Math. Mag.}
  \textbf{83},
  295--297.

\bibitem[Bailey \& Broadhurst(2000)]{bailey1}
  Bailey, D. H. \& Broadhurst, D. J. [2000]
  ``Parallel integer relation detection: techniques and applications,''
  {\it Math. Comput.}
  \textbf{70},
  1719--1736.

\bibitem[Bailey {\it et al.}(2006)]{bailey2}
  Bailey, D. H., Borwein, J. M., Kapour, V. \& Weisstein, E. W. [2006]
  ``Ten problems in experimental mathematics,''
  {\it Am. Math. Mon.}
  \textbf{113},
  481--509.

\bibitem[Kotsireas \& Karamanos(2004)]{kk1}
  Kotsireas, I. S. \& Karamanos, K. [2004]
  ``Exact computation of the bifurcation point $B_4$ of the logistic map
  and  the Bailey-Broadhurst conjectures,''
  {\it Int. J. Bifurcat. Chaos}
  \textbf{14},
  2417--2423.

\bibitem[Lewis(2008)]{lewis}
  Lewis, R. H. [2008]
  ``Heuristics to accelerate the Dixon resultant,''
  {\it Math. Comput. Simulat.}
  \textbf{77},
  400--407.

\bibitem[Mandelbrot(1980)]{mandelbrot}
  Mandelbrot, B. B. [1980]
  ``Fractal aspects of the iteration of $z \rightarrow \lambda z(1-z)$
      for complex $\lambda$ and $z$,''
  {\it Ann. NY Acad. Sci.}
  \textbf{357},
  249--259.

\bibitem[Hardy \& Wright(2008)]{hardy}
  Hardy, G. H. \& Wright, E. M. [2008]
  {\it An Introduction to the Theory of Numbers}
  (Oxford University Press, USA).

\bibitem[Riordan(1958)]{riordan}
  Riordan, J. [1958]
  {\it An Introduction to Combinatorial Analysis}
  (Wiley, New York).

\bibitem[Hao(2000)]{hao2}
  Hao, B.-L. [2000]
  ``Number of periodic orbits in continuous maps of
  the interval complete solution of the counting problem,''
  {\it Ann. Comb.}
  \textbf{4},
  339--346.

\bibitem[Lutzky(1988)]{lutzky}
  Lutzky, M. [1988]
  ``Counting stable cycles in unimodal iterations,''
  {\it Phys. Lett. A}
  \textbf{131},
  248--250.

\bibitem[Mira(1987)]{mira}
  Mira, C. [1987]
  {\it Chaotic Dynamics}
  (World Scientific, Singapore).

\bibitem[Blackhurst(2011)]{blackhurst}
  Blackhurst, J. [2011]
  ``Polynomials of the bifurcation points of the logistic map,''
  {\it Int. J. Bifurcat. Chaos}
  \textbf{21},
  1869--1877.

\bibitem[H\'enon(1976)]{henon}
  H\'enon, M. [1976]
  ``A two-dimensional mapping with a strange attractor,''
  {\it Commun. Math. Phys.}
  \textbf{50},
  69--77.

\bibitem[Huang(1985)]{huang}
  Huang, Y.-N. [1985]
  ``Determination of the stable periodic orbits for the H\'enon map by analytical method,''
  {\it Chin. Phys. Lett.}
  \textbf{2},
  97--100.

\bibitem[Hitzl(1985)]{hitzl}
  Hitzl, D. L. \& Zele, F. [1985]
  ``An exploration of the H\'enon quadratic map,''
  {\it Physica D}
  \textbf{14},
  305--326.

\bibitem[May(1979)]{may2}
  May, R. M. [1979]
  ``Bifurcations and dynamic complexity in ecological systems,''
  {\it Ann. N. Y. Acad. Sci.}
  \textbf{316},
  517--529.

\bibitem[Brown(1984a)]{brown3}
  Brown, A. [1984a]
  ``Solutions of period three for a non-linear difference equation,''
  {\it J. Austral. Math. Soc. Ser. B}
  \textbf{25},
  451--462.

\bibitem[Brown(1984b)]{brown4}
  Brown, A. [1984b]
  ``Solutions of period four for a non-linear difference equation,''
  {\it J. Austral. Math. Soc. Ser. B}
  \textbf{26},
  146--164.

\bibitem[Zeng(1985)]{zeng}
  Zeng, W. [1985]
  ``On the number of stable cycles in the cubic map,''
  {\it Commun. Theor. Phys.}
  \textbf{8}
  273--280.

\end{thebibliography}


%\end{multicols}

\end{document}


