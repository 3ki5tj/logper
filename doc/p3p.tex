\documentclass[12pt]{article}
\usepackage{amsmath}

\begin{document}
\title{Comments on the period three cycle}
\author{Cheng Zhang}
\maketitle

In this note, we shall
  find the bifurcation point of the period-3 cycle,
  and show the orbit is real after the onset.

\section{The bifurcation point}
Due to heavier algebra, we shall work with simplified map
\[
x_{n+1} = f(x_n) = R - {x_n}^2,
\]
and $a$, $b$ and $c$ denote three successive points in the period-3 orbit,
$b=f(a), c=f(b), a=f(c)$.
For $X \equiv a + b + c$, we have
\begin{align}
X^2 - X + 2 - R = 0.
\label{eq:xr}
\end{align}
%
Now
\begin{align}
 Y \equiv a b + b c + c a
= \frac{1}{2} [ (a + b + c)^2 - (a^2 + b^2 + c^2) ]
= X - R - 1,
\label{eq:y}
\end{align}
where we have used $(a^2 + b^2 + c^2) = 3R - (a+b+c)$ and \eqref{eq:xr}.

Similarly, we express $Z \equiv abc$ in terms of $X$ by
\begin{align}
  a^3 + b^3 + c^3 - 3 a b c =
 (a + b + c) (a^2 + b^2 + c^2 - a b - b c - c a)
\label{eq:pow3}
\end{align}
For the left hand side, we have $a^2 = R - b$, so $a^3 = R a - ab$, and
\begin{align}
  a^3 +b^3 + c^3
= R (a+b+c) - (ab+bc+ca)
= (R - 1) X + R + 1.
\label{eq:pow3a}
\end{align}
%
For the right hand side,
\begin{align}
& (a + b + c) (a^2 + b^2 + c^2 - a b - b c - c a)
= X [X^2 - 3(ab + bc + ca)] \notag\\
=& X (-2X + 4R + 1)
= (4R-1) X -2(R-2).
\label{eq:pow3b}
\end{align}
Plugging \eqref{eq:pow3a} and \eqref{eq:pow3b} to \eqref{eq:pow3}, we have
\begin{equation}
Z \equiv abc = -RX + R - 1.
\label{eq:abc}
\end{equation}
Solving \eqref{eq:abc} yields $X = 1 + (1+Z)/R$.
Plugging it into \eqref{eq:xr} yields
\begin{equation}
  R^3 - 2R^2 + (1+Z)R + (1+Z)^2 = 0.
\end{equation}

At the onset point $(-2a)(-2b)(-2c) = 1$, or $Z = -1/8$, we have
\[
  (R-7/4) (R^2 - R/4 + 7/16) = 0.
\]
The only real solution is $R = 7/4$, so $r = 1+\sqrt{1+4R} = 1+ \sqrt 8$.

At the bifurcation point $(-2a)(-2b)(-2c) = -1$, or $Z = 1/8$, we have
\[
  R^3 - 2 R^2 + 9R/8 - 81/64 = 0.
\]
The only real solution is
$
R = \frac{2}{3}
  +\frac{1}{4} \sqrt[3]{ \frac{1915}{54} + \frac{5}{2} \sqrt{201}}
  +\frac{1}{4} \sqrt[3]{ \frac{1915}{54} - \frac{5}{2} \sqrt{201}}.
$

\section{The existence of the real orbit}

Next we show that the orbit is real after the onset, i.e., $R>7/4$.
By definition $a$, $b$ and $c$ are solutions of
\[
  q^3 - X q^2 + Y q - Z = 0.
\]
The discriminant is
\[
\Delta = X^2 Y^2 - 4 Y^3 - 4 X^3 Z + 18 X Y Z - 27 Z^2
  = (4X^2 - 6X + 9)^2,
\]
where we have used \eqref{eq:y} and \eqref{eq:abc},
and eliminated $R$ by \eqref{eq:xr}.
Thus as long as $X$ is real, $\Delta > 0$, which means there are always three
real solutions.

\end{document}
