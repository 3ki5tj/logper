%\documentclass[12pt]{article}
\documentclass[twocolumn,prl]{revtex4-1}
\usepackage{amsmath}

\begin{document}
\title{On the 3-cycle of the logistic map}
\author{Cheng Zhang}
\date{April 15, 2012}
\begin{abstract}
We review a simple derivation of 
  the onset and bifurcation points the 3-cycle
  of the logistic map.
\end{abstract}
\maketitle


%\section{Introduction}
The logistic map reads
\[
  z_{n+1} = F(z_n) = r z_n (1 - z_n),
\]
We want to find what $r$ values permit a \emph{stable} 3-cycle,
i.e., a solution of $z_{n+3} = z_n$,
which can sustain a small deviation in the initial $z_0$.
The two boundary points $r_0$ and $r_1$ 
  of the permissible window $(r_0, r_1)$
  are called the \emph{onset} and \emph{bifurcation} points.
%
By a change of variables
\begin{equation}
  x_n = r (z_n - 1/2),
  R = (2 r - r^2)/4,
\label{eq:chvars}
\end{equation}
we get a simplified map
\[
x_{n+1} = f(x_n) = R - {x_n}^2.
\]
The onset and bifurcation points will be solved
  in terms of $R$, and the corresponding $r$ values 
  are obtained by \eqref{eq:chvars}.

%\section{Cyclic polynomials}

Let $a$, $b$ and $c$ be the three points in the 3-cycle:
\begin{equation}
  b=f(a), c=f(b), a=f(c).
\label{eq:abc}
\end{equation}
We define the cyclic polynomials
$X = a+b+c$,
$X_k = a^k+b^k+c^k$ (for $k = 2,3$),
$Y = ab+bc+ca$, and $Z = abc$.
We will show that the equations of $X$, $Y$, and $Z$
are more helpful in determining the two points of the 3-cycle.



%\section{Onset point}

For the onset point \cite{zhang},
we obtain from $c - b = f(b) - f(a)$ that
$c-b = -(b+a)(b-a)$;
Similar,
$a-c = -(c+b)(c-b)=(c+b)(b+a)(b-a)$.
Now $(b-a) + (c-b) + (a-c) = 0$ means
\[
1 - (b+a) + (c+b)(b+a) = 0.
\]
Cycling variables
$a\rightarrow b$,
$b\rightarrow c$,
$c\rightarrow a$,
and adding up the three distinct versions yields
\begin{equation}
3 - 2X + X^2 + Y = 0.
\label{eq:stair}
\end{equation}
%
On the other hand, the sum of \eqref{eq:abc} gives
$X = 3R - X_2$,
and $X^2 = X_2 + 2 Y$;
so
\begin{equation}
X = 3R - X^2 + 2Y.
\label{eq:XY}
\end{equation}
%
Eliminating $Y$ by \eqref{eq:stair} and \eqref{eq:XY}, we get
\begin{equation}
  X^2 - X + 2 - R = 0,
\label{eq:XR}
\end{equation}
This equation has a real root only if $2 - R \le 1/4$.
Thus, $R = 7/4$ (so $r = 1+\sqrt{1+4R} = 1+ \sqrt 8$)
is the onset for the only real 3-cycle.



%\section{Bifurcation point}

Let us express cyclic polynomials as linear functions of $X$.
Using \eqref{eq:stair} and \eqref{eq:XR}, we get
\begin{align}
  Y    &= X - R - 1, \notag \\
  X_2  &= X^2 - 2Y = 3R-X,
\label{eq:Y}
\end{align}
%
For $X_3$,
we have $a^3 = R a - ab$ from $b = f(a)$,
summing over cyclic versions yields,
\begin{equation}
X_3 = R X - Y = (R-1) X + R + 1.
\label{eq:X3}
\end{equation}
%
By the identity
$X_3 - 3 Z = X (X_2 - Y)$,
\eqref{eq:X3}, \eqref{eq:Y}, and \eqref{eq:XR},
we have
\begin{equation}
Z = -RX + R - 1.
\label{eq:Z}
\end{equation}
Now \eqref{eq:XR} can be rewritten in terms of $Z$:
\begin{equation}
  R^3 - 2R^2 + (1+Z)R + (1+Z)^2 = 0.
\label{eq:RZ}
\end{equation}

We now use the derivative condition:
at the onset (bifurcation) point,
$\frac{d}{da}f(f(f(a))) = +1$ ($-1$),
which by the chain rule, gives
$f'(c)f'(b)f'(a) = -8abc = \pm1$, or
\begin{equation}
  Z = \mp1/8.
\label{eq:Zder}
\end{equation}
%
At the onset point $Z = -1/8$, \eqref{eq:RZ} becomes
\[
  (R-7/4) (R^2 - R/4 + 7/16) = 0.
\]
The real solution $R = 7/4$ agrees with the previous solution.
%
At the bifurcation point $Z = 1/8$, we have
\[
  R^3 - 2 R^2 + 9R/8 - 81/64 = 0.
\]
The only real solution is
$
R = \frac{2}{3}
  +\frac{1}{4} \sqrt[3]{ \frac{1915}{54} + \frac{5}{2} \sqrt{201}}
  +\frac{1}{4} \sqrt[3]{ \frac{1915}{54} - \frac{5}{2} \sqrt{201}}.
$

We can also show $a$, $b$, $c$ are real after the onset ($R\ge7/4$).
By definition, $a$, $b$ and $c$ are the distinct roots of
\[
  x^3 - X x^2 + Y x - Z = 0.
\]
The discriminant is
\[
\Delta = X^2 Y^2 - 4 Y^3 - 4 X^3 Z + 18 X Y Z - 27 Z^2
  = (4X^2 - 6X + 9)^2,
\]
where we have used \eqref{eq:Y} and \eqref{eq:Z},
and eliminated $R$ by \eqref{eq:XR}.
Thus as long as $X$ is real, $\Delta > 0$, which means there are always three
real solutions.

\begin{thebibliography}{}
\bibitem{zhang} C. Zhang,
  Period three begins,
  \textit{Mathematics Magazine}
  \textbf{83},
  295-297
  (2010).
\end{thebibliography}

\end{document}

